

\chapter{قیمت}

یکی از مسائل مهم در طراحی محصول قیمت آن است. البته با توجه به این که این محصول به صورت نمونه اولیه طراحی شده است، طبیعتا قیمت تمام شده آن از محصولی که بخواهد تولید عمده بشود بالاتر خواهد بود. در جدول \ref{tab:1} قیمتی تخمین زده شده و هزینه نهایی پروژه آورده شده است.

عمده تفاوت قیمت بین اعداد تخمینی و عدد نهایی، به دلیل تغییر رزبری‌پای از 3B به 4، اضافه شدن آردوینو و همچنین اضافه شدن فشارسنج دستی صورت گرفته است.




\begin{table}
	\centering
	\begin{tabular}{|c|c|c|c|} 
		\hline
		ردیف & قطعه             & قیمت تخمینی & قیمت نهایی  \\ 
		\hline
		1    & \lr{MQ-135}           & 3.38       & 3.38        \\ 
		\hline
		2    & \lr{MPS20N004D}       & 5.26        & 0           \\ 
		\hline
		3    & \lr{DHT11 or KY-015}  & 45          & 45          \\ 
		\hline
		4    & \lr{MAX30205}         & 9.290       & 9.290       \\ 
		\hline
		5    & \lr{AD8232}           & 128         & 128         \\ 
		\hline
		6    & \lr{ECG Electrode}    & 96          & 96          \\ 
		\hline
		7    & \lr{MAX30102}         &3.163      & 3.163      \\ 
		\hline
		8    & \lr{Raspberry Pi LCD} & 1480        & 2000        \\ 
		\hline
		9    & \lr{Raspberry Pi 3B}  & 3000        & 0           \\ 
		\hline
		10   & \lr{Raspberry Pi 4}   & 0           & 4000        \\ 
		\hline
		11   & \lr{Arduino Uno}      & 0           & 400         \\ 
		\hline
		12   & \lr{Breadboard}       & 5.47        & 5.47        \\ 
		\hline
		13   & \lr{Wires}            & 100         & 100         \\ 
		\hline
		14   & \lr{Resistors}        & 7           & 10          \\ 
		\hline
		15   & \lr{ADS1115}          & 154         & 0           \\ 
		\hline
		16   & \lr{Sphygmomanometer} & 0           & 700         \\ 
		\hline
		17   & \lr{ECG Pads}         & 12          & 50          \\ 
		\hhline{|====|}
		18   & \lr{Total}            & 5.5588      & 8069        \\
		\hline
	\end{tabular}
\caption{جدول قیمت محصول (قیمت‌ها به واحد هزارتومان)}
\label{tab:1}
\end{table}

