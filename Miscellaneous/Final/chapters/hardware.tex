\section{طراحی و پیاده‌سازی سخت‌افزار}

اصلی‌ترین قسمت این پروژه، طراحی و پیاده‌سازی قسمت‌های سخت‌افزاری آن است. در زیر لیستی از قطعات سخت‌افزاری مورد استفاده آمده است و پس‌ از آن توضیحاتی در مورد هر یک از سنسور‌ها و نحوه کارکرد و راه‌اندازی آن ذکر شده است.


\begin{itemize}
	\item برد \lr{Raspberry Pi 4}
	\item برد \lr{Arduino UNO}
	\item صفحه نمایش لمسی ۷ اینچ مخصوص \lr{Raspberry Pi}
	\item سنسور آلودگی هوا \lr{MQ135}
	\item سنسور دما و رطوبت هوا \lr{DHT11}
	\item سنسور ضربان قلب و اکسیژن‌ خون \lr{Max30102}
	\item سنسور دمای بدن \lr{Max30205}
	\item سنسور نوار قلب \lr{AD8232}
	\item فشار سنج و گوشی پزشکی
\end{itemize}


\subsection{سنسورهای محیطی}

د, سنسور محیطی اصلی در این پروژه وجود دارند. سنسور \lr{MQ135} که وظیفه اندازه‌گیری آلودگی هوا را داشته و سنسور \lr{DHT11} که وظیفه اندازه‌گیری دما و رطوبت را دارد. سنسور آلودگی‌هوا به آردوینو متصل شده و سنسور اندازه‌گیری دما و رطوبت هوا مستقیما به رزبری‌پای متصل می‌شود.


\subsubsection{سنسور دما و رطوبت‌هوا}

سنسور مورد استفاده برای این بخش، \lr{DHT11} است که از قابلیت انتقال داده به صورت دیجیتال پشتیبانی کرده و برای همین به راحتی مطابق شکل \ref{fig:2} به رزبری‌پای متصل می‌شود.

\begin{figure}[h]
	\centering
	\includegraphics[width=0.5\textwidth]{figs/dht11.jpg}
	
	\caption{اتصال سنسور \lr{DHT11}}
	\label{fig:2}
\end{figure}


برای خواندن مقادیر از کتاب‌خانه‌ی \lr{Adafruit-Blinka}
\footnote{\lr{https://pypi.org/project/Adafruit-Blinka/}}
استفاده شده است. این کتاب‌خانه با مشخص کردن پین متصل به سنسور، به راحتی امکان خواندن دما و رطوبت هوا را به ما می‌دهد.

کد اصلی مربوط به این قسمت در زیر آورده شده است:

\begin{latin}
\begin{lstlisting}[language=python]
dht11_sensor = adafruit_dht.DHT11(board.D23)
temp = dht11_sensor.temperature
humidity = dht11_sensor.humidity

\end{lstlisting}
\end{latin}

همچنین در شکل \ref{fig:13} پین‌های ماژول \lr{DHT11} مشخص شده‌اند.

\begin{figure}[h]
	\centering
	\includegraphics[width=0.5\textwidth]{figs/dht11-2.png}
	
	\caption{پین‌های سنسور \lr{DHT11}}
	\label{fig:13}
\end{figure}

همان طور که مشخص است، این سنسور دو پین \lr{GND} و \lr{VCC} برای تغذیه داشته و یک پین \lr{DATA} برای انتقال داده‌ها به صورت دیجیتالی دارد.

\subsubsection{سنسور آلودگی‌هوا}

با توجه به این که  سنسور \lr{MQ135} خروجی اصلی خود را به صورت آنالوگ تحویل داده و حتی رابط \lr{I2C} هم ندارد، آن را به برد آردوینو متصل کرده و از طریق اتصال رزبری‌پای به آردوینو با پورت \lr{USB}، کد مربوط به آن را از طریق رزبری به برد آردوینو انتقال داده و داده‌های لازم را دریافت می‌کنیم.

نحوه اتصال این سنسور در کنار سنسور قبلی در شکل \ref{fig:3} قرار دارد.

\begin{figure}[ht!]
	\centering
	\includegraphics[width=0.3\textwidth]{figs/dhtmq2.jpg}
	
	\caption{اتصال سنسور \lr{MQ135}}
	\label{fig:3}
\end{figure}


در کد آردویینو، مقادیر مربوط به این سنسور هر ۵ ثانیه خوانده می‌شود:
\begin{latin}
	\begin{lstlisting}[language=python]
#include <Wire.h>

void setup() {
	Serial.begin(9600);
	Wire.begin();
}
int counter = 0; // 1 milisecond

void loop() {	
	if (counter % 5000 == 0) // 5 second
	{
		int pollution = analogRead(A0);
		Serial.print("pollution,");
		Serial.println(pollution);
	} 
	counter += 10;
	delay(10);
}
	\end{lstlisting}
\end{latin}

سپس در کد پایتون روی رزبری، این مقادیر روی یک فایل ریخته می‌شود و پس از آن  مقادیر باقی سنسورها به سرور ارسال می‌گردد.


همچنین در شکل \ref{fig:14} پین‌های ماژول \lr{MQ135}، مشخص شده‌اند.

\begin{figure}[h]
	\centering
	\includegraphics[width=0.5\textwidth]{figs/mq135.jpg}
	
	\caption{پین‌های ماژول سنسور \lr{MQ135}}
	\label{fig:14}
\end{figure}

همان طور که مشخص است، این سنسور دو پین \lr{GND} و \lr{VCC} برای تغذیه داشته و دو پین دیگر برای انتقال آنالوگ و دیجیتال دارد. نکته قابل توجه این است که پین دیجتیالی فقط به صورت صفر و یکی و برای مشخص کردن گذر مقدار گاز موجود از حد مجاز کاربرد دارد و نمی‌تواند مقدار دقیق آن را انتقال بدهد و مقدار دقیق آن تنها از طریق پین آنالوگ قابل دریافت است. برای تنظیم حد مجاز، از پتانسیومتری که در قسمت بالا سمت راست \ref{fig:14} قرار دارد استفاده می‌شود.



\subsection{سنسورهای بدن}

سه سنسور اصلی برای علائم مربوط به بدن انسان در این پروژه وجود دارند. سنسور دمای بدن \lr{Max30205}، سنسور  ضربان قلب و اکسیژن خون \lr{Max30102} و سنسور نوار قلب یعنی \lr{AD8232}. سنسور \lr{Max30102} مستقیما به رزبری‌پای متصل می‌شود ولی دو سنسور دیگر از طریق آردوینو با رزبری‌پای ارتباط برقرار می‌کنند.


\subsubsection{سنسور دمای بدن}



برای سنجش دمای بدن از سنسور \lr{Max30205} استفاده شده است. با تماس انگشت به آن، بعد از مدتی دمای سنسور با دمای انگشت همدما شده و دمای بدن را نشان خواهد داد. در صورت عدم تماس هم می‌توان از آن برای مشاهده دمای محیط استفاده کرد.

این سسنور به آردوینو متصل شده و از طریق اتصال آردوینو به رزبری‌پای، اطلاعات آن را مشاهده می‌کنیم. در شکل \ref{fig:4} تصویر اتصال این سنسور قرار دارد.

\begin{figure}[h]
	\centering
	\includegraphics[width=0.5\textwidth]{figs/max30205.jpg}
	
	\caption{اتصال سنسور \lr{Max30205}}
	\label{fig:4}
\end{figure}


برای خواندن مقادیر این سنسور از کتاب‌خانه‌ی\lr{Protocentra\_MAX30205}
\footnote{\lr{https://github.com/Protocentral/Protocentral\_MAX30205}}
 استفاده شده که امکان خواندن دمای بدن را از طریق آردویینو می‌دهد.

\begin{latin}
	\begin{lstlisting}[language=python]
#include <Wire.h>
#include "Protocentral_MAX30205.h"
MAX30205 tempSensor;

void setup() {
	Serial.begin(9600);
	Wire.begin();
	
	while(!tempSensor.scanAvailableSensors()){
		Serial.println("Couldn't find the sensor." );
		delay(3000);
	}
}
int counter = 0; // 1 milisecond

void loop() {	
	if (counter % 5000 == 0) // 5 second
	{
		float temp = tempSensor.getTemperature(); 
		Serial.print("temp,");
		Serial.println(temp ,2);
	}
	counter += 10;
	delay(10);
}

\end{lstlisting}
\end{latin}

همچنین در شکل \ref{fig:15} پین‌های ماژول \lr{MAX30205} مشخص شده‌اند.

\begin{figure}[h]
	\centering
	\includegraphics[width=0.3\textwidth]{figs/MAX30205-2.jpg}
	
	\caption{پین‌های سنسور \lr{MAX30205}}
	\label{fig:15}
\end{figure}


دو پین \lr{GND} و \lr{VDD} برای تغذیه مدار استفاده می‌شوند. پین‌های \lr{SCL} و \lr{SDA} پین‌هایی هستند که در تمامی سنسور‌هایی که به صورت \lr{I2C} کار می‌کنند، مشاهده می‌کنیم. \lr{SCL} برای تامین کلاک استفاده شده و انتقال داده از طریق \lr{SDA} صورت می‌گیرد. پین‌های \lr{A0,A1,A2}
برای تعیین شماره آدرس \lr{I2C} مدار استفاده می‌شوند و بسته به مقادیری که به آنان داده بشود، آدرس‌ مشخصی برای \lr{I2C} در مداری که سنسور به آن متصل می‌شود مشخص می‌شود. پین \lr{OS} مخفف 
\lr{Overtemprature Shutdown}
است. این پین به نوعی مانند یک ترموستات عمل می‌کند و وقتی دمای اندازه‌گیری شده، از مقداری که درون یکی از رجیستر‌های درونی مدار مشخص شده است بالاتر برود، مقدار آن یک می‌شود. برای کاربرد‌های رایج توصیه سازنده این است که این پین یا اصلا متصل نشده و یا به مدار \lr{PullUp} با مقاومت $4.7k\Omega$ متصل شود.

\subsubsection{سنسور ضربان قلب و اکسیژن خون}


برای این بخش از سنسور \lr{Max30102} استفاده شده است. این سنسور با کمک دو چراغ کوچک قرمز و مادون قرمز، ضربان قلب و درصد اشباع اکسیژن در خون (\lr{SpO2}) را اندازه‌گیری می‌کند. این سنسور بدون مشکل از طریق \lr{I2C} به رزبری‌پای متصل می‌شود. نحوه اتصال این سنسور را در شکل \ref{fig:5} مشاهده می‌کنید.

\begin{figure}[h]
	\centering
	\includegraphics[width=0.5\textwidth]{figs/max30102.jpg}
	
	\caption{اتصال سنسور \lr{Max30102}}
	\label{fig:5}
\end{figure}


برای راه‌اندازی این سنسور از کدهای مخزن ‌متن‌باز \lr{doug-burrell/max30102}
\footnote{\lr{https://github.com/doug-burrell/max30102}}
با اندکی تغییرات
 استفاده شده است. این مخزن با خواندن مقادیر سنسورهای قرمز/مادون‌قرمز و پردازش آن‌ها، مقادیر ضربان قلب و اکسیژن خون را به طور دقیق محاسبه می‌کند. کدهای مربوط به این مخزن در فایل‌های \lr{heartrate\_monitor.py}
 و
 \lr{max30102.py}
 و 
 \lr{hrcalc.py}
 قرار دارند.
از امکانات این کتاب‌خانه به صورت زیر در کد رزبری استفاده شده است:

\begin{latin}
	\begin{lstlisting}[language=python]
from heartrate_monitor import HeartRateMonitor

hrm = HeartRateMonitor(print_raw=False, print_result=False)
hrm.start_sensor()

bpm = hrm.bpm
spo2 = hrm.spo2

\end{lstlisting}
\end{latin}


	همچنین در شکل \ref{fig:16} پین‌های ماژول \lr{MAX30102} مشخص شده‌اند.
	
	\begin{figure}[h]
		\centering
		\includegraphics[width=0.3\textwidth]{figs/MAX30102-2.jpg}
		
		\caption{پین‌های سنسور \lr{MAX30102}}
		\label{fig:16}
	\end{figure}

دو پین \lr{GND} و \lr{VIN} برای تغذیه مدار استفاده می‌شوند. پین‌های \lr{SCL} و \lr{SDA} پین‌هایی هستند که در تمامی سنسور‌هایی که به صورت \lr{I2C} کار می‌کنند، مشاهده می‌کنیم. \lr{SCL} برای تامین کلاک استفاده شده و انتقال داده از طریق \lr{SDA} صورت می‌گیرد. پین‌های \lr{RD} و \lr{IRD} برای این هستند که به صورت دستی، \lr{LED} های قرمز و مادون قرمز ماژول را روشن کرد. در حالت کلی نیازی به وصل کردن سیمی به این دو سنسور نیست. پین \lr{INT} در زمان‌هایی که در سنسور‌های درونی مدار \lr{Interrupt} رخ بدهد به حالت \lr{LOW} وارد می‌شود و در بقیه زمان‌ها \lr{HIGH} است. \lr{Interrupt} ها می‌توانند شامل مواردی نظیر پر شدن صف \lr{FIFO} درونی سنسور، تاثیر گذاشتن نور زمینه محیط بر نتیجه در اثر \lr{Overflow} واحد \lr{Ambient Light Cancellation} و موارد دیگر رخ بدهند. در کاربرد عمومی نیازی به وصل کردن این پین به خروجی خاصی نیست.

\subsubsection{سنسور نوار قلب}


از سنسور \lr{AD8232} برای بدست‌ آوردن \lr{ECG} (نوار قلب) استفاده می‌شود. همراه این سنسور بسته \lr{Lead} سه تایی اتصال به بدن وجود دارد. به این \lr{Lead} ها باید پد‌های مخصوص متصل شده و به بدن متصل شوند. برای استفاده از سنسور،‌ آن را به آردوینو متصل می‌کنیم. Lead های‌ آن هم باید مطابق شکل \ref{fig:6} به بدن بیمار متصل بشوند.

\begin{figure}[h]
	\centering
	\includegraphics[width=0.25\textwidth]{figs/ecg.jpg}
	
	\caption{نحوه اتصال Lead‌ های سنسور نوار قلب به بدن}
	\label{fig:6}
\end{figure}



همچنین نحوه اتصال این سنسور به همراه دو سنسور دیگر بدن در شکل \ref{fig:7} قرار دارد.


\begin{figure}[h]
	\centering
	\includegraphics[width=0.5\textwidth]{figs/allbody.jpg}
	
	\caption{اتصال سنسورهای بدن به رزبری‌پای و آردوینو در کنار هم}
	\label{fig:7}
\end{figure}


کدهای آردویینو مربوط به این سنسور در زیر آمده است. هر ۱۰ میلی‌ثانیه مقادیر مربوط به این سنسور از طریق آردویینو خوانده می‌شود.

\begin{latin}
	\begin{lstlisting}[language=python]
#include <Wire.h>
void setup() {
	Serial.begin(9600);
	Wire.begin();
	pinMode(10, INPUT);
	pinMode(11, OUTPUT);
}
int counter = 0; // 1 milisecond
void loop() {
	if (counter % 10 == 0) // 10 milisecond
	{
		Serial.print("ecg,");
		if(digitalRead(10) == 1 || digitalRead(11) == 1) {
		Serial.println('!');
		}
		else {
		int ecg = analogRead(A1);
		Serial.println(ecg);
		}  
	}
	counter += 10;
	delay(10);
}

	\end{lstlisting}
\end{latin}


همچنین در شکل \ref{fig:17} پین‌های ماژول \lr{AD8232} مشخص شده‌اند.

\begin{figure}[h]
	\centering
	\includegraphics[width=0.3\textwidth]{figs/ad8232.jpg}
	
	\caption{پین‌های سنسور \lr{AD8232}}
	\label{fig:17}
\end{figure}


دو پین \lr{GND} و \lr{3.3V} برای تغذیه مدار استفاده می‌شوند.  سه پین \lr{RA,LA,RL} برای گرفتن خروجی \lr{Lead}های متصل به بدن به صورت جداگانه استفاده می‌شوند. در این پروژه از آن‌ها استفاده‌ای نشده است. در عوض از جک \lr{3.5} میلی‌متری برای اتصال \lr{Lead}ها استفاده شده است. پین \lr{OUT} خروجی آنالوگ مدار است که با استفاده از آپ‌امپ، تقویت و کنترل شده است. پین‌های \lr{LOD-}
و
\lr{LOD+}
به معنی \lr{Leads of Detection} هستند. پین \lr{LOD-} زمانی که سنسور نتواند اتصال درست الکترود منفی به بدن را تشخیص بدهد به حالت \lr{HIGH} می‌رود. پین \lr{LOD+} زمانی که سنسور نتواند حضور الکترود مثبت را تشخیص بدهد به حالت \lr{HIGH} می‌رود. در صورتی که هر دو \lr{LOW} باشند، یعنی \lr{Lead}ها به درستی متصل شده‌اند.
پین \lr{SDN}، به معنی \lr{Shutdown Control Input} است و به صورت \lr{Active Low} کار می‌کند. این پین در مدارهای سیستم‌های نهفته که تنها قرار است گاهی اوقات قابلیت اندازه‌گیری ضربان قلبشان فعال شده و در سایر زمان‌ها غیرفعال است استفاده می‌شود. در صورتی که ورودی \lr{LOW} به این پین داده بشود، عملا بخش عمده ماژول از کار می‌افتد و مدار جریان خیلی کمی از منبع تغذیه کشیده و مصرف کمی پیدا می‌کند. در نتیجه در سیستم‌هایی که مصرف توان اهمیت بالایی دارد و از طرفی قصد داشته باشیم امکان کنترل الکترونیکی فعال یا غیرفعال بودن سنسور را داشته باشیم، می‌توان از طریق این پین کل سنسور را به حالت خاموش وارد کرد.

\subsubsection{انداز‌ه‌گیری فشار خون }

با توجه به این که اندازه‌گیری فشار خون به شکل دقیق معمولا توسط فرد آموزش دیده انجام می‌گیرد و ساخت فشار سنج اتوماتیک از حوزه کاری این محصول خارج بوده و خود یک محصول جداگانه است، فشار در این محصول به شکل انسانی با استفاده از فشارسنج اندازه‌گیری شده و در منویی که برای ثبت فشار در نرم‌افزار قرار گرفته است، وارد می‌شود. تصویر فشارسنج در شکل \ref{fig:12} قرار گرفته است.



\begin{figure}[ht!]
	\centering
	\includegraphics[width=0.4\textwidth]{figs/cuff.jpg}
	
	\caption{فشارسنج}
	\label{fig:12}
\end{figure}

\subsection{کدهای سخت‌افزار}

در قسمت‌های بالا، برای هر سنسور کدهای مهم آن توضیح داده شده است. در این قسمت به طور خلاصه به ساختار فایل‌هایی که برای کدهای سخت‌افزار وجود دارند پرداخته می‌شود.



\definecolor{folderbg}{RGB}{124,166,198}
\definecolor{folderborder}{RGB}{110,144,169}

\def\Size{4pt}
\tikzset{
	folder/.pic={
		\filldraw[draw=folderborder,top color=folderbg!50,bottom color=folderbg]
		(-1.05*\Size,0.2\Size+5pt) rectangle ++(.75*\Size,-0.2\Size-5pt);  
		\filldraw[draw=folderborder,top color=folderbg!50,bottom color=folderbg]
		(-1.15*\Size,-\Size) rectangle (1.15*\Size,\Size);
	}
}
\begin{figure}[ht!]
	
\begin{latin}
	
	\begin{forest}
		for tree={
			font=\ttfamily,
			grow'=0,
			child anchor=west,
			parent anchor=south,
			anchor=west,
			calign=first,
			inner xsep=7pt,
			edge path={
				\noexpand\path [draw, \forestoption{edge}]
				(!u.south west) +(7.5pt,0) |- (.child anchor) pic {folder} \forestoption{edge label};
			},
			% style for your file node 
			file/.style={edge path={\noexpand\path [draw, \forestoption{edge}]
					(!u.south west) +(7.5pt,0) |- (.child anchor) \forestoption{edge label};},
				inner xsep=2pt,font=\small\ttfamily
			},
			before typesetting nodes={
				if n=1
				{insert before={[,phantom]}}
				{}
			},
			fit=band,
			before computing xy={l=15pt},
		}  
		[Hardware
		[arduino
		[main
		[main.ino ,file]]
		]
		[heartrate
		[heartrate\_monitor.py,file
		]
		[hrcalc.py , file]
		[max30102.py,file]
		]
		[transmission
		[send\_values.py,file]
		[store\_arduino\_values.py,file]
		]
		[utils
		[utils.py , file]
		]
		[main.py , file]
		[patient\_id.txt , file]
		[requirements.txt , file]
		[README.md,file]
		]
	\end{forest}

\end{latin}
	\caption{ساختار پوشه‌بندی و فایل‌های مربوط به سخت‌افزار}
\label{fig:hardware}
\end{figure}

کد موجود در پوشه \verb+arduino+، باید از طریق نرم‌افزار آردوینو به روی آردوینو منتقل شود. این کد برای سنسورهایی که به آردوینو متصل هستند استفاده شده و داده‌های آنان را دریافت می‌کنند.

کدهای موجود در پوشه \verb+heartrate+ کدهای پایتونی هستند که مربوط به اندازه‌گیری ضربان قلب و اکسیژن خون می‌شوند. در فایل \verb+max30102+ اطلاعات لازم برای ارتباط برقرار کردن با ماژول قرار گرفته و مستقیما از کدهای مخزن متن‌باز \lr{doug-burrell/max30102} با اندکی تغییرات گرفته شده است. فایل \verb+hrcalc+ برای انجام یکسری محاسبات براساس داده‌های دریافتی و تبدیل آن‌ها به ضربان قلب است. +\verb+heartrate_monitor+ فایل نهایی است که داده‌ها را با استفاده از دو فایل دیگر جمع‌آوری می‌کند.

کدهای موجود در پوشه \verb+transmission+ برای دریافت و ارسال فایل‌ها هستند. کدهای موجود در \verb+store_arduino_values+ از طریق پورت \lr{USB} با آردوینو ارتباط برقرار کرده و اطلاعات دریافتی از سمت‌ آن‌را روی فایل می‌نویسد. فایل \verb+send_values+ اطلاعاتی که مستقیما از طریق رزبری قابل دریافت هستند را دریافت کرده و همچنین از اطلاعات ذخیره شده توسط \verb+store_arduino_values+ استفاده می‌کند تا تمامی اطلاعات جمع‌آوری شده و به سرور ارسال بشوند.

کد موجود در پوشه \verb+utils+ شامل یکسری داده‌های سراسری مورد استفاده برنامه نظیر آدرس سرور، محل ذخیره‌سازی فایل‌ها، فعال بودن حالت \lr{DEBUG} و... است.

فایل \verb+patient_id.txt+ تنها شامل شماره آی‌دی بیمار خواهد بود. از آن برای ارسال اطلاعات به سرور استفاده می‌شود.

فایل \verb+main.py+ کد اصلی برای اجرای برنامه است که کارهای لازم برای اجرای کل برنامه را در خود دارد و از بقیه فایل‌ها استفاده می‌کند و در نهایت این فایل باید اجرا بشود.
 
فایل \verb+requirements.txt+ شامل لیست پکیج‌های مورد نیاز پایتونی برای اجرای برنامه است.

فایل \verb+README.md+ شامل اطلاعاتی در مورد نحوه اجرا کردن برنامه است.




\subsection{بسته بندی}

برای بسته‌بندی، با توجه به هماهنگی‌های صورت گرفته قرار است که از پرینتر سه‌بعدی استفاده شود. از این رو قسمت بسته‌بندی محصول هنوز کامل نیست. در شکل \ref{fig:9} طرح آماده شده برای بسته‌بندی را مشاهده می‌کنید.

\begin{figure}[h]
	\centering
	\includegraphics[width=0.6\textwidth]{figs/package.png}
	
	\caption{بسته‌بندی طراحی شده در نرم‌افزار \lr{Tinkercad}}
	\label{fig:9}
\end{figure}

وضعیت فعلی محصول به همراه تمامی سنسور‌ها در شکل \ref{fig:10} آمده است.


\begin{figure}[h]
	\centering
	\includegraphics[width=1.0\textwidth]{figs/all.jpg}
	
	\caption{محصول به همراه تمامی سنسور‌ها }
	\label{fig:10}
\end{figure}

