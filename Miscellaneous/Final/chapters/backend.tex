\section{طراحی و پیاده‌سازی سرور}

سرور در این پروژه وظیفه دریافت اطلاعات از رزبری‌پای،‌ ذخیره آن‌ها در پایگاه داده و تحویل دادن آن‌ها با فیلترها و به شکل مناسب به نرم‌افزار طراحی شده را دارد.

برای طراحی سرور از فریم‌ورک \lr{Django} و \lr{Django Rest Framework} استفاده شده است که به ما امکان طراحی سریع و در عین حال اصولی  سروری که برای این پروژه مناسب باشد را می‌داد. برای پایگاه‌داده هم از \lr{MariaDB} استفاده شده است که براساس \lr{MySQL} توسعه یافته است و با قواعد \lr{MySQL} هماهنگی کامل دارد. برای راحتی کار توسعه سرور و نصب نیازمندی‌های آن، از Docker و \lr{Docker Compose} استفاده شده است تا به راحتی همه اجزای مختلف سرور مستقل از سیستمی که توسعه روی آن صورت می‌گیرد به شکل مناسب و سریع استقرار یابد.

\lr{}همچنین برای این که نیاز به استفاده از دستورات \lr{Docker} کمینه شود، یک فایل  \lr{Makefile} هم نوشته شده است که با دو دستور اصلی \lr{make build} و \lr{make up}، راه‌اندازی اولیه سرور و بالا آوردن آن بعد از راه‌اندازی اولیه قابل انجام است. همچنین با دستور \lr{make down} می‌توان سرور را خاموش کرد.



سرور پروژه از دو قسمت اصلی تشکیل شده است. بخش Patients که در آن مدل مربوط به بیماران (شامل نام و نام‌خانوادگی،‌ شماره تلفن، قد، وزن و جنسیت) در فایل \lr{models.py} تعریف شده است. همچنین با استفاده از کلاس‌های \lr{Django Rest Framework} تمامی کار‌های مربوط به ایجاد، آپدیت، حذف و دریافت اطلاعات هویتی به بیماران در این قسمت و فایل \lr{views.py} انجام می‌شود.

بخش دیگر Records است. در این جا مدل اصلی برای ذخیره داده‌های بیماران و همچنین داده‌های محیطی تعریف شده است. همچنین در قسمت \lr{views.py} علاوه بر تعریف دو کلاس براساس Django Rest Framework که امکان عملیات‌های یجاد، آپدیت، حذف و دریافت را برای داده‌های بیماران فراهم کنند، سه تابع برای دریافت آخرین داده، دریافت آخرین داده‌ها از زمان مشخص شده و دریافت داده‌ها با تعیین فیلتر‌های مختلف (زمان، بیمار و سنسورهای خاص) تعریف شده است. همچنین تابعی که آخرین داده را برای ما فراهم می‌کند، برای هر فیلد مستقلا به دنبال آخرین داده می‌گردد تا اگر در داده‌هایی که برای آخرین زمان ثبت شده‌اند، برای برخی از سنسورها داده‌ای وجود نداشت، آخرین داده موجود گزارش شود. در صورتی که هیچ داده‌ای وجود نداشته باشد، مقادیر پیش‌فرض ثبت شده برای آن فیلد گزارش می‌شوند.

علاوه بر این، با استفاده از ابزار Swagger مستندات خودکار برای استفاده از API ها هم ایجاد شده است. در صورتی که سرور به صورت محلی اجرا شود، این مستندات از آدرس
\lr{localhost:8000/swagger/}
در دسترس خواهند بود.
 تصویر آن را در شکل \ref{fig:8}
مشاهده می‌کنید.

\begin{figure}[h]
	\centering
	\includegraphics[width=1.0\textwidth]{figs/swagger.png}
	
	\caption{مستندات \lr{Swagger}}
	\label{fig:8}
\end{figure}


