\documentclass[12pt]{article}
\usepackage{graphicx,import}
\usepackage{float}
\usepackage[svgnames]{xcolor} 
\usepackage{makecell}
\usepackage{fancyhdr}
\usepackage{subcaption}
\usepackage{hyperref}
\usepackage{enumitem}
\usepackage{cite}
\usepackage[many]{tcolorbox}
\usepackage{listings }
\usepackage[a4paper, total={6in, 8in} , bottom = 25mm , top = 25mm, headheight = 1.25cm , includehead,includefoot,heightrounded ]{geometry}
\usepackage{afterpage}
\usepackage{amssymb}
\usepackage{pdflscape}
\usepackage{gensymb}
\usepackage{textcomp}
\usepackage{tikz,pgfplots}
\usepackage{xecolor}
\usepackage{rotating}
\usepackage{pdfpages}
\usepackage[Kashida]{xepersian}
\usepackage[T1]{fontenc}
\usepackage{tikz}
\usepackage[utf8]{inputenc}
\usepackage{PTSerif} 
\usepackage{seqsplit}
\usepackage{hhline}
\usepackage{pgfgantt}
\usepackage{graphicx}

\graphicspath{ {./images/} }

\renewcommand\theadalign{bc}
\renewcommand\theadfont{\bfseries}
\renewcommand\theadgape{\Gape[4pt]}
\renewcommand\cellgape{\Gape[4pt]}

\usepackage[edges]{forest}

\usepackage{listings}
\usepackage{xcolor}

\hypersetup{
	colorlinks   = true, %Colours links instead of ugly boxes
	urlcolor     = blue, %Colour for external hyperlinks
	linkcolor    = blue, %Colour of internal links
	citecolor   = red %Colour of citations
}
 
\definecolor{codegreen}{rgb}{0,0.6,0}
\definecolor{codegray}{rgb}{0.5,0.5,0.5}
\definecolor{codepurple}{rgb}{0.58,0,0.82}
\definecolor{backcolour}{rgb}{0.95,0.95,0.92}
 
\NewDocumentCommand{\codeword}{v}{
\texttt{\textcolor{blue}{#1}}
}
\lstset{language=java,keywordstyle={\bfseries \color{blue}}}


\lstdefinestyle{mystyle}{
    backgroundcolor=\color{backcolour},   
    commentstyle=\color{codegreen},
    keywordstyle=\color{magenta},
    numberstyle=\tiny\color{codegray},
    stringstyle=\color{codepurple},
    basicstyle=\ttfamily\normalsize,
    breakatwhitespace=false,         
    breaklines=true,                 
    captionpos=b,                    
    keepspaces=true,                 
    numbers=left,                    
    numbersep=5pt,                  
    showspaces=false,                
    showstringspaces=false,
    showtabs=false,                  
    tabsize=2
}

\lstset{style=mystyle}

\settextfont[Scale=1.2 ,BoldFont={Bahij Nazanin-Bold} , ItalicFont = {IRNazanin}]{Bahij Nazanin-Regular}
\setlatintextfont[Scale = 1.0]{Garamond}
\DefaultMathsDigits 
\DeclareMathSizes{11}{19}{13}{9} 
%\DeclareMathSizes{12}{14.4}{8}{9}





\newenvironment{changemargin}[2]{%
\begin{list}{}{%
\setlength{\topsep}{0pt}%
\setlength{\leftmargin}{#1}%
\setlength{\rightmargin}{#2}%
\setlength{\listparindent}{\parindent}%
\setlength{\itemindent}{\parindent}%
\setlength{\parsep}{\parskip}%
}%
\item[]}{\end{list}}


\definecolor{foldercolor}{RGB}{124,166,198}

\tikzset{pics/folder/.style={code={%
    \node[inner sep=0pt, minimum size=#1](-foldericon){};
    \node[folder style, inner sep=0pt, minimum width=0.3*#1, minimum height=0.6*#1, above right, xshift=0.05*#1] at (-foldericon.west){};
    \node[folder style, inner sep=0pt, minimum size=#1] at (-foldericon.center){};}
    },
    pics/folder/.default={20pt},
    folder style/.style={draw=foldercolor!80!black,top color=foldercolor!40,bottom color=foldercolor}
}

\forestset{is file/.style={edge path'/.expanded={%
        ([xshift=\forestregister{folder indent}]!u.parent anchor) |- (.child anchor)},
        inner sep=1pt},
    this folder size/.style={edge path'/.expanded={%
        ([xshift=\forestregister{folder indent}]!u.parent anchor) |- (.child anchor) pic[solid]{folder=#1}}, inner xsep=0.6*#1},
    folder tree indent/.style={before computing xy={l=#1}},
    folder icons/.style={folder, this folder size=#1, folder tree indent=3*#1},
    folder icons/.default={12pt},
}

\begin{document}


%%% title pages
\begin{titlepage}
\begin{center}
        
\vspace*{0.7cm}

\includegraphics[width=0.4\textwidth]{sharif1.png}\\
\vspace{0.5cm}
\textbf{ \Huge{\emph ‌آزمایشگاه سخت‌افزار} }\\
\vspace{0.5cm}
\textbf{ \Large{ پروپوزال پروژه} }
\vspace{0.2cm}
       
 
      \large \textbf{دانشکده مهندسی کامپیوتر}\\\vspace{0.2cm}
    \large   دانشگاه صنعتی شریف\\\vspace{0.2cm}
       \large   ﻧﯿﻢ سال دوم 01-00 \\\vspace{0.2cm}
      \noindent\rule[1ex]{\linewidth}{1pt}
استاد:\\
    \textbf{{جناب آقای دکتر اجلالی}}


دستیار آموزشی:\\
\textbf{{جناب آقای دکتر فصحتی}}

    \vspace{0.25cm}
    
    موضوع پروژه:\\
    
    \textbf{{نمایشگر علائم حیاتی بیمار (پروژه شماره ۱۴)}}
    
    \vspace{0.35cm}
    
    
        شماره گروه:
    \textbf{{۲}}\\
    
اعضای گروه:\\

    \textbf{{علیرضا تاج‌میرریاحی - 97101372}}
    \\
   
     \textbf{{امیرمهدی نامجو - 97107212}}   
   \\
   
    \textbf{{ صبا هاشمی  - 97100581}}
\end{center}
\end{titlepage}
%%% title pages


%%% header of pages
\newpage
\pagestyle{fancy}
\fancyhf{}
\fancyfoot{}
\cfoot{\thepage}
\chead{}
\rhead{\includegraphics[width=0.1\textwidth]{sharif.png}}
\lhead{پروپوزال پروژه}
%%% header of pages

\newfontfamily\terminal{Courier New Bold}

\KashidaOff
 \newcommand{\inlineLatin}[1]{
	\small{\lr{{\terminal #1}}}
}


\tableofcontents
\listoffigures
\listoftables

\section{مقدمه}


هدف از این پروژه، طراحی سیستم نمایشگر هوشمند علائم حیاتی بیمار و شرایط محیطی است. هسته این سیستم که از رزبری پای تشکیل شده است، اطلاعات حیاتی بیمار شامل دمای بدن، فشار خون، ضربان قلب، اکسیژن خون و نوار قلب (\lr{ECG}) را از طریق سنسور‌های مربوطه از بیمار دریافت کرده و در کنار آن، اطلاعات محیطی نظیر دما،‌ رطوبت و میزان آلودگی هوا را هم از طریق سنسورهایی دیگر دریافت می‌کند.

برای ارتباط برقرار کردن با این داده‌ها، رابط کاربری برای نمایشگر رزبری پای و همچنین رابط کاربری موبایل تهیه خواهد شد که پرستار یا پزشک از طریق آن بتوانند به داده‌های بیمار دسترسی پیدا کنند. همچنین این داده‌ها در فواصل زمانی معین به یک سرور ارسال شده و برای آنالیزهای بعدی و همچنین حفظ تاریخچه بیماران، در دیتابیس ذخیره می‌شوند.



\begin{table}[H]
	\centering
	\begin{tabular}{|l|l|l|} 
		\hline
		\makecell{\textbf{ردیف}} & \makecell{\textbf{ویژگی}}                                         & \makecell{\textbf{توضیحات}}                                                                                         \\ 
		\hhline{|===|}
		۱             & \makecell{\textbf{ثبت علائم حیاتی}}                               &
		\makecell{
		 علائم حیاتی کاربر شامل اکسیژن خون،\\ فشار خون، ضربان قلب و \lr{ECG} توسط سنسورها\\ دریافت و ثبت می‌شود.}           \\ 
		\hline
		۲             & \makecell{ \textbf{ثبت شرایط محیطی}    }                           & \makecell{
		 شرایط محیطی کاربر شامل دما، رطوبت\\ و میزان آلودگی هوا توسط سنسورها\\ دریافت و ثبت می‌شود.}                   \\ 
		\hline
		
		۳             & \makecell{\textbf{نمایش وضعیت کاربر و}\\  \textbf{ شرایط محیط روی صفحه نمایش}}
		  &
		 \makecell{
		  علائم حیاتی کاربر و شرایط محیطی روی نمایشگر\\ به طور زنده نشان داده می‌شود.}                                \\ 
		\hline
		۴             &
		 \makecell{\textbf{اتصال به اپلیکیشن}}                             &
		\makecell{
		 با استفاده از یک اپلیکیشن موبایل می‌توان\\ به دستگاه متصل شد و وضعیت کاربر\\ را از طریق موبایل مانیتور کرد.}  \\ 
		\hline
		۵             &
		\makecell{\textbf{مشاهده ارتباط علائم حیاتی و} \\ \textbf{شرایط محیطی}}     &
		\makecell{
		 در اپلیکیشن موبایل می‌توان نمودارهای\\ وضعیت کاربر و شرایط محیطی را در کنار هم\\ مشاهده و بررسی کرد.}         \\ 
		\hline
		۶             & \makecell{\textbf{تاریخچه استفاده}}                               &
		\makecell{
		 امکان مشاهده‌ی تاریخچه‌ی مقادیر سنسورها\\ در اپلیکیشن یا پنل ادمین وجود دارد.}                              \\
		\hline
	\end{tabular}
\caption{\label{features}جدول ویژگی‌های اصلی محصول}
\end{table}



\newpage
\section{روش انجام پروژه}

\subsection{بررسی کلی پروژه}
در این پروژه از برد رزبری پای $3$ استفاده خواهد شد. سنسورهای مربوط به بدن و هم‌چنین سنسورهای محیطی به این برد متصل می‌شوند و از طریق رزبری اطلاعات آن‌ها پردازش و از طریق وای‌فای به سرور ارسال می‌شود. 

سرور با استفاده از زبان پایتون و فریم‌ورک \lr{Django} زده می‌شود. وظیفه‌ی آن این است که اطلاعات را از رزبری دریافت و در دیتابیس ذخیره کند. دیتابیس مورداستفاده \lr{MySQL} خواهد بود. هم‌چنین سرور API های لازم برای دریافت اطلاعات را نیز در اختیار اپلیکیشن موبایل قرار می‌دهد.

در اپلیکیشن موبایل علائم حیاتی کاربر و شرایط محیطی در لحظه‌ی فعلی و هم چنین در طول یک بازه 

قابل مشاهده خواهد بود. برای پیاده‌سازی اپلیکیشن از فریم‌ورک \lr{React Native} استفاده می‌شود.

هم چنین یک نمایشگر نیز به برد رزبری متصل خواهد بود و مقادیر ثبت شده توسط سنسورها را نشان می‌دهد. با توجه به \lr{Multiplatform} بودن فریم‌ورک 
\lr{React Native}
و امکان استفاده آن بر روی دسکتاپ و وب، برای رابط کاربری این قسمت نیز از همین فریم‌ورک استفاده خواهد شد.


\subsection{سنسورها}

\subsubsection{دما و رطوبت}
برای ثبت دما و رطوبت از سنسور \lr{KY-015} استفاده می‌کنیم که چیپست مورد استفاده در آن \lr{DHT11} است. 

دقت این سنسور برای دما $\pm2$ درجه سانتی‌گراد و دقت آن برای رطوبت $\pm 5\% \text{RH}$ است. نرخ نمونه‌برداری آن نیز هر $2$ ثانیه یک بار است. این دقت و این نرخ نمونه‌برداری نیازی که ما در این پروژه داریم را رفع می‌کند و برای همین نیازی به هزینه‌ی بیشتر برای سنسورهای با نرخ نمونه‌برداری یا دقت بیشتر نیست.

\begin{figure}[h]
	\begin{center}
		\includegraphics[width=0.3\textwidth]{KY-015}
	\end{center}
	\caption{\lr{KY-015}}
\end{figure}


پروتوکل ارتباطی این سنسور تک سیم است و مقادیر دما و رطوبت را به طور دیجیتال انتقال می‌دهد و نیاز به قطعه‌‌ی دیگری برای 
اتصال به رزبری ندارد و به شکل زیر به رزبری متصل می‌شود:


\begin{figure}[h]
	\begin{center}
		\includegraphics[width=0.3\textwidth]{KY-015-raspberry}
	\end{center}
	\caption{اتصال \lr{KY-015} به رزبری}
\end{figure}


\subsubsection{ECG}

برای مانیتور وضعیت قلب از قطعه‌ی \lr{AD8232} استفاده می‌کنیم. این قطعه فعالیت الکتریکی قلب را اندازه‌گیری می‌کند و خروجی آن می‌تواند به صورت مقادیر آنالوگ یا برای رسم \lr{ECG} مورد استفاده قرار بگیرد. در نتیجه در کنار این قطعه نیاز به یک مبدل آنالوگ به دیجیتال داریم. 


\begin{figure}[h]
	\begin{center}
		\includegraphics[width=0.3\textwidth]{AD8232}
	\end{center}
	\caption{\lr{AD8232}}
\end{figure}

هم‌چنین برای استفاده و تست آن نیاز به سه الکترود و تعدادی پد برای اتصال به بدن داریم.



\subsubsection{آلودگی هوا}

برای ثبت آلودگی هوا، از سنسور \lr{MQ135} استفاده می‌کنیم. این سنسور برای تشخیص گازهای آمونیاک، اکسید‌های نیتروژن، الکل، کربن‌ دی‌اکسید، بنزن و دود حاصل از سوختن در هوا استفاده می‌شود. حساسیت این سنسور از طریق پتانسیومتری که روی آن تعبیه شده است قابل اندازه گیری است. این قطعه دو خروجی دیجیتال و آنالوگ دارد. خروجی دیجیتال آن صرفا در صورتی که یکی از این گاز‌ها از حد خاصی بیش‌تر بشود،‌ به حالت فعال و در غیر این صورت به حالت غیرفعال در می‌آید. خروجی آنالوگ آن بسته به مقاومت تنظیم شده و مطابق نمودار مشخصی، ولتاژی را متناسب با سطح گازهای موجود در هوا گزارش می‌کند. برای استفاده از این قطعه به مبدل آنالوگ به دیجیتال نیاز داریم.

\begin{figure}[h]
	\begin{center}
		\includegraphics[width=0.3\textwidth]{mq-135}
	\end{center}
	\caption{\lr{MQ135}}
\end{figure}

لازم به ذکر است که این قطعه در محدوده دمایی $-10$ تا $45$ درجه سانتی‌گراد و تا حداکثر سطح رطوبت 
$95 \% \text{RH}$
کارایی دارد. با این حال باید توجه کرد که نمودار‌های مربوط به آن برای دمای $20 \pm 2$ درجه و سطح رطوبت
$65 \% \text{RH} \pm 5\%$
تهیه شده است و در صورت استفاده در شرایط دیگر،‌ نیاز به اندکی محاسبات برای جلوگیری از کاهش دقت وجود دارد.

\begin{figure}[h]
	\begin{center}
		\includegraphics[width=0.3\textwidth]{mq-135-raspberry.png}
	\end{center}
	\caption{اتصال \lr{MQ135} به رزبری}
\end{figure}




\subsubsection{ضربان قلب و اکسیژن خون}

\begin{figure}[h]
	\begin{center}
		\includegraphics[width=0.3\textwidth]{max30102}
	\end{center}
	\caption{\lr{MAX30102}}
\end{figure}

برای اندازه‌گیری ضربان قلب و اکسیژن خون، از سنسور 
\lr{MAX30102}
استفاده خواهیم کرد. این قطعه متناسب استفاده در دستگاه‌های پوشیدنی نظیر مچ‌بند‌های ورزشی و همچنین اسمارت‌فون‌ها طراحی شده است و در نتیجه از دقت قابل قبولی برخوردار است. این سنسور از طریق پروتکل 
\lr{I2C}
امکان ارتباط با میکروکنترلرها و همچنین پردازنده‌های ARM از جمله رزبری‌پای را دارد و از این لحاظ برای کار ما مناسب است.



دقت مبدل آنالوگ به دیجیتال این قطعه $18$ بیت بوده که دقت مناسبی است. بازه دمایی قابل استفاده آن هم $-40$ تا $85$ درجه سانتی‌گراد است.

\subsubsection{مبدل آنالوگ به دیجیتال}
برای تبدیل سیگنال‌های آنالوگ به دیجیتال از ماژول
\lr{ADS1115}
استفاده خواهیم کرد. این ماژول دارای $4$ کانال خروجی است و $16$ بیت دقت دارد. با کمک این ماژول سیگنال‌های آنالوگ را به فرم دیجیتال تبدیل می‌کنیم. راه‌اندازی و کار با این ماژول کتابخانه تحت آردوینو توسعه یافته است؛ اما می‌توان از آن در بردهای
\lr{Raspberry Pi}
نیز به عنوان مبدل آنالوگ به دیجیتال، استفاده کرد. نحوه‌ی اتصال این ماژول به رزبری مطابق تصویر زیر است:

\begin{figure}[h]
	\begin{center}
		\includegraphics[width=0.3\textwidth]{ADS1115_Raspberry}
	\end{center}
	\caption{اتصال ADS1115 به رزبری}
\end{figure}

\subsubsection{فشار خون}

\begin{figure}[h]
	\begin{center}
		\includegraphics[width=0.3\textwidth]{MPS20N0040D}
	\end{center}
	\caption{\lr{MPS20N0040D}}
\end{figure}

برای اندازه‌گیری فشار، از قطعه‌ی
\lr{MPS20N0040D}
استفاده می‌شود. این ماژول یک ماژول حسگر فشار و ارتفاع است داده را دریافت و به تراشه‌ی
\lr{HX710}
- که یک مبدل آنالوگ به دیجیتال است - انتقال می‌دهد. از این ماژول برای سنجش فشار محیط استفاده می‌شود. این ماژول از ارتباط
\lr{I2C}
پشتیبانی نمی‌کند.


از سایر کاربردهای این حسگر می‌توان مانیتور کردن فشار خون (هدف ما از استفاده)، فشار چرخ خودرو و مانیتور فشار محیط اشاره کرد.

\subsubsection{دمای بدن}
برای اندازه‌گیری دمای بدن، از سنسور
\lr{MAX30205}
استفاده می‌کنیم. سنسور دمای
\lr{MAX30205}
به‌درستی و با دقت بالا
$(0.1^\circ C)$
دما را اندازه می‌گیرد و همچنین یک خروجی آلارم/وقفه
\lr{(Interrupt)}
/خاموشی در دمای بالاتر از حد مجاز ارائه می‌دهد (دقت این ماژول استاندارد کلینیکال
\lr{ASTM E1112}
را رعایت می‌کند).
این سنسور دمای اندازه‌گیری شده را با یک مبدل آنالوگ به دیجیتال
\lr{(ADC)}
با دقت بالا به دیجیتال تبدیل می‌کند.

\begin{figure}[h]
	\begin{center}
		\includegraphics[width=0.3\textwidth]{MAX30205}
	\end{center}
	\caption{\lr{MAX30205}}
\end{figure}

همچنین سنسور
\lr{MAX30205}
از طریق درگاه سریال
\lr{I2C}
با انواع میکروکنترلر ارتباط برقرار می‌کند. این واسط سریال امکان ارسال و دریافت استاندارد بایت‌ها را برای خواندن داده‌ی دما و تنظیم رفتار خروجی دمای بیش از حد فراهم می‌‌سازد.

\begin{figure}[h]
	\begin{center}
		\includegraphics[width=0.2\textwidth]{max30102-raspberry}
	\end{center}
	\caption{اتصال \lr{MAX30102} به رزبری}
\end{figure}


\newpage
\subsection{معماری سیستم}

نمودار معماری سیستم در شکل زیر آورده شده است.


\begin{figure}[h]
	\begin{center}
		\includegraphics[width=0.9\textwidth,trim={0 10cm 0 0},clip]{drawio}
	\end{center}
	\caption{معماری سیستم}
\end{figure}
\newpage
\section{جدول هزینه‌ها}

در زیر جدول هزینه‌های تخمینی پروژه آورده شده است.



\begin{table}[h]
	\centering
	\begin{tabular}{|c|c|c|c|c|c|} 
		\hline
		\textbf{ردیف} & \textbf{قطعه}  &
		\makecell{
			\textbf{فی}\\
		 \textbf{ (هزارتومان)}
	} & \textbf{تعداد} & 
		\makecell{
			\textbf{قیمت کل}\\
		\textbf{ (هزارتومان)} 
	}
		& \makecell{\makecell{\textbf{لینک} \\ \textbf{ فروشنده}}}  \\ 
		\hhline{|======:b|}
		
		1             &
		\makecell{
		 سنسور کیفیت\\ و آلودگی هوا \lr{MQ-135}      
		}
		  & \lr{38.3}                    & \lr{1}              &                \lr{38.3}              &         \href{https://thecaferobot.com/store/mq-135-air-quality-sensor-module-1}{\textcolor{blue}{\underline{{لینک}}}}                                    \\ 
		\hline
		2             & 
		\makecell{
		سنسور اندازه گیری\\ فشار خون \lr{MPS20N0040D}
	}
		 & \lr{26.5}                    & \lr{1}              &             \lr{26.5}                 &        \href{https://thecaferobot.com/store/pressure-meter-mps20n0040d}{\textcolor{blue}{\underline{{لینک}}}}                                      \\ 
		\hline
		3             & سنسور دما و رطوبت هوا \lr{KY-015}           & \lr{45}                       & 
		\lr{1}              &              \lr{45}           &          
		\makecell{\href{https://www.google.com}{\textcolor{blue}{\underline{{لینک}}}} \\ \href{https://thecaferobot.com/store/temprature-sensor-dht11-dfrobot-1}{\textcolor{blue}{\underline{{محصول}}}} \\ \href{https://thecaferobot.com/store/temprature-sensor-dht11-dfrobot-1}{\textcolor{blue}{\underline{{مشابه}}}}}
		\\ 
		\hline
		4             & سنسور دمای بدن \lr{MAX30205} &
		 \lr{290.9}                   & \lr{1}              &          \lr{290.9}                   &       \href{https://shop.aftabrayaneh.com/cjmcu_max30205_human_body_temperature_sensor-aftab.html}{\textcolor{blue}{\underline{{لینک}}}}                                       \\ 
		\hline
		5             & سنسور \lr{ECG} ضربان قلب~ \lr{AD8232}            &
		 \lr{128.0}                   & \lr{1}              &                  \lr{128.0}           &    \href{https://thecaferobot.com/store/heart-rate-monitor-ad8232-ecg}{\textcolor{blue}{\underline{{لینک}}}}                                          \\ 
		\hline
		6             & الکترود \lr{ECG} 
		                           &\lr{ 96.0}                    & \lr{1}              &     \lr{96}                         &\href{https://thecaferobot.com/store/ecg-heart-rate-ekg-muscle-electric-emg-sensor-accessories-headphone-plug-cable}{\textcolor{blue}{\underline{{لینک}}}}                                              \\ 
		\hline
		7             & ماژول اکسیمتر \lr{MAX30102}                 & 
		\lr{163.3}                   & \lr{1}              &            \lr{163.3}                  &     \href{https://daneshjookit.com/module/sensor/%D8%A8%DB%8C%D9%88%D9%85%D8%AA%D8%B1%DB%8C%DA%A9/3574-pulse-oximeter-max30102-module.html}{\textcolor{blue}{\underline{{لینک}}}}  
			                                       \\ 
		\hline
		8             & صفحه نمایش \lr{LCD} رزبری‌پای               & \lr{1480}                       & \lr{1}              &               \lr{1480}               &                                              \href{https://www.digikala.com/product/dkp-7886676/%D9%85%D8%A7%DA%98%D9%88%D9%84-%D9%86%D9%85%D8%A7%DB%8C%D8%B4%DA%AF%D8%B1-%D9%84%D9%85%D8%B3%DB%8C-%D8%B1%D8%B2%D8%A8%D8%B1%DB%8C-%D9%BE%D8%A7%DB%8C-%D9%85%D8%AF%D9%84-hdmi-%DA%A9%D8%AF-5/}{\textcolor{blue}{\underline{{لینک}}}} \\ 
		\hline
		9             & رزبری پای \lr{3b}                            & \lr{3000}                       & 
		\lr{1}              &                      \lr{3000}        
		&\href{https://www.digikala.com/product/dkp-288904/%D8%A8%D8%B1%D8%AF-%D8%B1%D8%B3%D9%BE%D8%A8%D8%B1%DB%8C-%D9%BE%D8%A7%DB%8C3-%D9%85%D8%AF%D9%84-b/}{\textcolor{blue}{\underline{{لینک}}}}  
		                                            \\ 
		\hline
		10            & بردبورد                                & \lr{23.75}                       & \lr{2}              &                     \lr{47.5}        &                                             \href{https://www.digikala.com/product/dkp-7199135/%D8%A8%D8%B1%D8%AF-%D8%A8%D9%88%D8%B1%D8%AF-%D9%85%D8%AF%D9%84-400s-%D8%A8%D8%B3%D8%AA%D9%87-2-%D8%B9%D8%AF%D8%AF%DB%8C/}{\textcolor{blue}{\underline{{لینک}}}} \\ 
		\hline
		11            & سیم جامپر                              & \lr{1}                       & \lr{100}            &                              \lr{100}&        
		                                      \href{https://www.digikala.com/product/dkp-4548386/%D8%B3%DB%8C%D9%85-%D8%AC%D8%A7%D9%85%D9%BE%D8%B1-%D9%85%D8%AF%D9%84ff30-%D8%A8%D8%B3%D8%AA%D9%87-40-%D8%B9%D8%AF%D8%AF%DB%8C/}{\textcolor{blue}{\underline{{لینک}}}} \\ 
		\hline
		12            & مقاومت                                 & \lr{0.14}                       & \lr{50}             &                 \lr{7}             &       \href{https://www.digikala.com/product/dkp-2299913/%D9%85%D9%82%D8%A7%D9%88%D9%85%D8%AA-10-%DA%A9%DB%8C%D9%84%D9%88-%D8%A7%D9%87%D9%85-%DA%A9%D8%AF-10k-%D8%A8%D8%B3%D8%AA%D9%87-50-%D8%B9%D8%AF%D8%AF%DB%8C/}{\textcolor{blue}{\underline{{لینک}}}}        
		                               \\ 
		\hline
		13            & 
		 مبدل آنالوگ به دیجیتال \lr{ADS1115}         & \lr{154.0}                   & \lr{1}              &                              \lr{154}&                              \href{https://daneshjookit.com/module/useful/%D9%85%D8%A8%D8%AF%D9%84-%D9%88-%D8%B1%DA%AF%D9%88%D9%84%D8%A7%D8%AA%D9%88%D8%B1/2136-adc-module-ads1115.html}{\textcolor{blue}{\underline{{لینک}}}}                \\ 
		\hline
		14            & پد \lr{ECG}        
		                         & \lr{2}                       & \lr{6}              & \lr{12}                           & \href{https://thecaferobot.com/store/ecg-and-emg-accessories-electrode-sheet}{\textcolor{blue}{\underline{{لینک}}}}                                             \\ 
		\hline
		& \textbf{مجموع}                         &                         &                &      \lr{5588.5}                        &                                              \\
		\hline
	\end{tabular}
\caption{برآورد هزینه‌ها}
\end{table}



\newpage
\section{زمان‌بندی}

\subsection{چارت زمانی}

\begin{figure}[H]
	
	\begin{center}
		\begin{ganttchart}[
			expand chart=1\textwidth,
		    vrule label font=\tiny,
			title label font=\tiny, 
			bar label font=\tiny, 
			y unit title=1cm,
			y unit chart=0.8cm,
			x unit=1cm,
			vgrid,hgrid, 
			title label anchor/.style={below=-1.6ex},
			title left shift=0,
			title right shift=0,
			title height=1,
			progress label text={},
			bar height=0.6,
			group right shift=0,
			group top shift=.5,
			group height=.2]{3}{16}
			%labels
			\gantttitle{اسفند}{2}
			\gantttitle{فروردین}{4}
			\gantttitle{اردی‌بهشت}{4}
			\gantttitle{خرداد}{4}
			\\

			\gantttitle{\rl{هفته ۳}}{1} 			
			\gantttitle{\rl{هفته ۴}}{1} 
			\gantttitle{\rl{هفته ۱}}{1} 
			\gantttitle{\rl{هفته ۲}}{1} 
			\gantttitle{\rl{هفته ۳}}{1} 
			\gantttitle{\rl{هفته ۴}}{1} 
			\gantttitle{\rl{هفته ۱}}{1} 
			\gantttitle{\rl{هفته ۲}}{1} 
			\gantttitle{\rl{هفته ۳}}{1} 
			\gantttitle{\rl{هفته ۴}}{1} 
			\gantttitle{\rl{هفته ۱}}{1} 
			\gantttitle{\rl{هفته ۲}}{1} 
			\gantttitle{\rl{هفته ۳}}{1} 
			\gantttitle{\rl{هفته ۴}}{1} 

			%
			\\
			%tasks
			\ganttbar[progress=0]{\rl{نهایی کردن پروپوزال}}{3}{4} \\
			\ganttbar[progress=0]{\rl{تهیه‌ی قطعات}}{4}{6} \\
			\ganttbar[progress=0]{\rl{آشنایی با رزبری}}{5}{6} \\
			\ganttbar[progress=0]{\rl{تهیه معماری کامل سیستم}}{5}{6} \\
			
			\ganttgroup{\rl{سرور}}{6}{13} \\
			\ganttbar[progress=0]{\rl{بیسیک سرور}}{6}{7} \\	
			\ganttbar[progress=0]{\rl{اتصال رزبری و سرور}}{8}{9} \\
			\ganttbar[progress=0]{\rl{تکمیل سرور}}{10}{13} \\
			
			\ganttgroup{\rl{اپ موبایل}}{6}{13} \\
			\ganttbar[progress=0]{\rl{بیسیک اپ موبایل}}{6}{7} \\
			\ganttbar[progress=0]{\rl{اتصال اپ موبایل به سرور}}{8}{8} \\
			\ganttbar[progress=0]{\rl{رسم نمودارها و قابلیت‌های اضافه}}{9}{13} \\

			\ganttgroup{\rl{سخت‌افزار}}{7}{12} \\
			\ganttbar[progress=0]{\rl{اتصال و تست صفحه نمایش}}{7}{8} \\
			\ganttbar[progress=0]{\rl{اتصال و تست سنسورهای محیطی}}{7}{8} \\
			\ganttbar[progress=0]{\rl{اتصال و تست سنسورهای بدن}}{9}{12} \\
			
			\ganttbar[progress=0]{\rl{تست کلی سیستم}}{13}{14} \\
			
			\ganttbar[progress=0]{\rl{بررسی بازخوردها}}{15}{15}
			
			%relations 
			\ganttlink{elem0}{elem1} 
			\ganttlink{elem0}{elem3} 
			\ganttlink{elem0}{elem4} 
			\ganttlink{elem0}{elem8} 
			\ganttlink{elem1}{elem12}
			\ganttlink{elem3}{elem12} 			 
			\ganttlink{elem4}{elem16} 
			\ganttlink{elem8}{elem16}
			\ganttlink{elem12}{elem16} 
			\ganttlink{elem17}{elem17} 
			
			\ganttlink{elem5}{elem6} 
			\ganttlink{elem6}{elem7} 

			\ganttlink{elem5}{elem10}	
			
			\ganttlink{elem9}{elem10}	
			\ganttlink{elem10}{elem11} 
 
			\ganttvrule[vrule/.append style={blue, thin},vrule offset=1]{\rl{پروپوزال}}{4}
			\ganttvrule[vrule/.append style={blue, thin},vrule offset=1]{\rl{گزارش اول}}{6}
			\ganttvrule[vrule/.append style={blue, thin},vrule offset=1]{\rl{گزارش دوم}}{8}
			\ganttvrule[vrule/.append style={blue, thin},vrule offset=1]{\rl{گزارش سوم}}{10}
			\ganttvrule[vrule/.append style={blue, thin},vrule offset=1]{\rl{گزارش چهارم}}{12}	
			\ganttvrule[vrule/.append style={blue, thin},vrule offset=1]{\rl{گزارش اولیه}}{14}
			\ganttvrule[vrule/.append style={blue, thin},
vrule offset=1
]{\rl{گزارش نهایی}}{15}
		\end{ganttchart}
	\end{center}
	\caption{گانت چارت پروژه}
	
\end{figure}
	

\newpage
	\subsection{فعالیت‌های برنامه‌ریزی شده برای انتهای هر فاز}
	
	\subsubsection*{فاز اول - ۱۷ فروردین ۱۴۰۱}
	
	\begin{itemize}
		\item تهیه قطعات
		\item  ارائه معماری سیستم
	\end{itemize}
	
	\subsubsection*{فاز دوم - ۳۱ فروردین ۱۴۰۱}
	
	\begin{itemize}
		\item  اتصال و تست صفحه نمایش
		\item  اتصال و تست سنسورهای محیطی
	\end{itemize}
	
	\subsubsection*{فاز سوم - ۱۴ اردی‌بهشت ۱۴۰۱}
	
	\begin{itemize}
		\item ورژن اولیه سرور
		\item ورژن اولیه اپلیکیشن موبایل
		\item  اتصال سرور، موبایل و رزبری به یک دیگر
	\end{itemize}
	
	\subsubsection*{فاز چهارم - ۲۸ اردی‌بهشت ۱۴۰۱}
	
		
	\begin{itemize}
		\item اتصال و تست سنسورهای بدن انسان 
	\end{itemize}

	\subsubsection*{فاز پنجم - ۱۱ خرداد ۱۴۰۱}
	
		
	\begin{itemize}
		\item ورژن نهایی سرور
		\item ورژن نهایی اپلیکیشن موبایل
		\item  تست نهایی سیستم
	\end{itemize}
	
	\subsubsection*{فاز ششم - ۱۸ خرداد ۱۴۰۱}
	
			
	\begin{itemize}
		\item رفع مشکلات ذکر شده در بازخورد 
	\end{itemize}

\end{document}



