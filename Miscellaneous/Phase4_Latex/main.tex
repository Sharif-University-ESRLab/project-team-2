\documentclass[12pt]{article}
\usepackage{graphicx,import}
\usepackage{float}
\usepackage[svgnames]{xcolor} 
\usepackage{makecell}
\usepackage{fancyhdr}
\usepackage{subcaption}
\usepackage{hyperref}
\usepackage{enumitem}
\usepackage[many]{tcolorbox}
\usepackage{listings }
\usepackage[a4paper, total={6in, 8in} , bottom = 25mm , top = 25mm, headheight = 1.25cm , includehead,includefoot,heightrounded ]{geometry}
\usepackage{afterpage}
\usepackage{amssymb}
\usepackage{pdflscape}
\usepackage{gensymb}
\usepackage{textcomp}
\usepackage{tikz,pgfplots}
\usepackage{xecolor}
\usepackage{rotating}
\usepackage{pdfpages}
\usepackage[Kashida]{xepersian}
\usepackage[T1]{fontenc}
\usepackage{tikz}
\usepackage[utf8]{inputenc}
\usepackage{PTSerif} 
\usepackage{seqsplit}
\usepackage{hhline}
\usepackage{pgfgantt}
\usepackage{graphicx}
\usepackage{filecontents}
\usepackage{url} % for "\url" macro
\usepackage{babel}
\usepackage[backend=bibtex, style=numeric]{biblatex}


\graphicspath{ {./images/} }

\renewcommand\theadalign{bc}
\renewcommand\theadfont{\bfseries}
\renewcommand\theadgape{\Gape[4pt]}
\renewcommand\cellgape{\Gape[4pt]}

\usepackage[edges]{forest}

\usepackage{listings}
\usepackage{xcolor}

\hypersetup{
	colorlinks   = true, %Colours links instead of ugly boxes
	urlcolor     = blue, %Colour for external hyperlinks
	linkcolor    = blue, %Colour of internal links
	citecolor   = red %Colour of citations
}
 
\definecolor{codegreen}{rgb}{0,0.6,0}
\definecolor{codegray}{rgb}{0.5,0.5,0.5}
\definecolor{codepurple}{rgb}{0.58,0,0.82}
\definecolor{backcolour}{rgb}{0.95,0.95,0.92}
 
\NewDocumentCommand{\codeword}{v}{
\texttt{\textcolor{blue}{#1}}
}
\lstset{language=java,keywordstyle={\bfseries \color{blue}}}


\lstdefinestyle{mystyle}{
    backgroundcolor=\color{backcolour},   
    commentstyle=\color{codegreen},
    keywordstyle=\color{magenta},
    numberstyle=\tiny\color{codegray},
    stringstyle=\color{codepurple},
    basicstyle=\ttfamily\normalsize,
    breakatwhitespace=false,         
    breaklines=true,                 
    captionpos=b,                    
    keepspaces=true,                 
    numbers=left,                    
    numbersep=5pt,                  
    showspaces=false,                
    showstringspaces=false,
    showtabs=false,                  
    tabsize=2
}

\lstset{style=mystyle}

\settextfont[Scale=1.2 ,BoldFont={Bahij Nazanin-Bold} , ItalicFont = {IRNazanin}]{Bahij Nazanin-Regular}
\setlatintextfont[Scale = 1.0]{Garamond}
\DefaultMathsDigits 
\DeclareMathSizes{11}{19}{13}{9} 
%\DeclareMathSizes{12}{14.4}{8}{9}




\newenvironment{changemargin}[2]{%
\begin{list}{}{%
\setlength{\topsep}{0pt}%
\setlength{\leftmargin}{#1}%
\setlength{\rightmargin}{#2}%
\setlength{\listparindent}{\parindent}%
\setlength{\itemindent}{\parindent}%
\setlength{\parsep}{\parskip}%
}%
\item[]}{\end{list}}


\definecolor{foldercolor}{RGB}{124,166,198}

\tikzset{pics/folder/.style={code={%
    \node[inner sep=0pt, minimum size=#1](-foldericon){};
    \node[folder style, inner sep=0pt, minimum width=0.3*#1, minimum height=0.6*#1, above right, xshift=0.05*#1] at (-foldericon.west){};
    \node[folder style, inner sep=0pt, minimum size=#1] at (-foldericon.center){};}
    },
    pics/folder/.default={20pt},
    folder style/.style={draw=foldercolor!80!black,top color=foldercolor!40,bottom color=foldercolor}
}

\forestset{is file/.style={edge path'/.expanded={%
        ([xshift=\forestregister{folder indent}]!u.parent anchor) |- (.child anchor)},
        inner sep=1pt},
    this folder size/.style={edge path'/.expanded={%
        ([xshift=\forestregister{folder indent}]!u.parent anchor) |- (.child anchor) pic[solid]{folder=#1}}, inner xsep=0.6*#1},
    folder tree indent/.style={before computing xy={l=#1}},
    folder icons/.style={folder, this folder size=#1, folder tree indent=3*#1},
    folder icons/.default={12pt},
}

\begin{document}


%%% title pages
\begin{titlepage}
\begin{center}
        
\vspace*{0.7cm}

\includegraphics[width=0.4\textwidth]{sharif1.png}\\
\vspace{0.5cm}
\textbf{ \Huge{\emph ‌آزمایشگاه سخت‌افزار} }\\
\vspace{0.5cm}
\textbf{ \Large{گزارش فاز چهارم} }
\vspace{0.2cm}
       
 
      \large \textbf{دانشکده مهندسی کامپیوتر}\\\vspace{0.2cm}
    \large   دانشگاه صنعتی شریف\\\vspace{0.2cm}
       \large   ﻧﯿﻢ سال دوم 01-00 \\\vspace{0.2cm}
      \noindent\rule[1ex]{\linewidth}{1pt}
استاد:\\
    \textbf{{جناب آقای دکتر اجلالی}}


دستیار آموزشی:\\
\textbf{{جناب آقای دکتر فصحتی}}

    \vspace{0.25cm}
    
    موضوع پروژه:\\
    
    \textbf{{نمایشگر علائم حیاتی بیمار (پروژه شماره ۱۴)}}
    
    \vspace{0.35cm}
    
    
        شماره گروه:
    \textbf{{۲}}\\
    
اعضای گروه:\\

    \textbf{{علیرضا تاج‌میرریاحی - 97101372}}
    \\
   
     \textbf{{امیرمهدی نامجو - 97107212}}   
   \\
   
    \textbf{{ صبا هاشمی - 97100581}}
\end{center}
\end{titlepage}
%%% title pages


%%% header of pages
\newpage
\pagestyle{fancy}
\fancyhf{}
\fancyfoot{}
\cfoot{\thepage}
\chead{}
\rhead{\includegraphics[width=0.1\textwidth]{sharif.png}}
\lhead{گزارش فاز چهارم}
%%% header of pages

\newfontfamily\terminal{Courier New Bold}

\KashidaOff
 \newcommand{\inlineLatin}[1]{
	\small{\lr{{\terminal #1}}}
}


\tableofcontents
\listoffigures

\newpage
\section{مقدمه}


محصول نهایی این پروژه، یک سیستم نمایشگر هوشمند علائم حیاتی بیمار و شرایط محیطی است. هسته این سیستم که از رزبری پای تشکیل شده است، اطلاعات حیاتی بیمار شامل دمای بدن، فشار خون، ضربان قلب، اکسیژن خون و نوار قلب (\lr{ECG}) را از طریق سنسور‌های مربوطه از بیمار دریافت کرده و در کنار آن، اطلاعات محیطی نظیر دما،‌ رطوبت و میزان آلودگی هوا را هم از طریق سنسورهایی دیگر دریافت می‌کند.

طبق زمان‌بندی ارائه شده در بخش \ref{gantt}، اقدامات مربوط به فاز چهارم پروژه عبارت‌اند از تکمیل اتصال و تست سنسورهای بدن، و کامل‌تر کردن سرور و نرم‌افزار موبایل. 


\section{گزارش انجام پروژه}
\subsection{سخت‌افزار}

در این بخش، به پیشرفت‌ها و چالش‌های زمینه راه‌اندازی قسمت‌های سخت‌افزاری پروژه، شامل تکمیل اتصال و تست سنسورهای بدن \ref{body} می‌پردازیم.

\subsubsection{ اتصال و تست سنسورهای بدن در کنار سنسورهای محیطی} \label{body}

در فازهای قبل هر کدام از سنسورهای زیر را به طور جداگانه راه‌اندازی کرده بودیم. در این فاز همه‌ی سنسورها شامل سنسورهای بدن و سنسورهای محیطی را به رزبری متصل کرده و کد تجمیع شده برای جمع‌آوری اطلاعات از همه‌ی سنسورها را پیاده سازی کردیم.

در کل پنج سنسور زیر به طور هم‌زمان راه‌اندازی شدند. جزئیات مربوط به هر کدام در مستندهای فازهای قبل موجود است.

\begin{enumerate}
	\item 
	سنسور \lr{MAX30102}: این سنسور برای سنجش اکسیژن خون و ضربان قلب است.
	
	\item 
	سنسور \lr{MAX30205}: این سنسور برای سنجش دمای بدن استفاده می‌شود.
	
	\item 
	سنسور \lr{Ad8232}: این سنسور برای ECG استفاده می‌شود.
	
	\item 
	سنسور \lr{DHT11}: این سنسور، هم دما و هم رطوبت هوا را اندازه‌گیری می‌کند.
	
	\item 
	سنسور \lr{MQ135}: ا این سنسور برای اندازه‌گیری آلودگی هوا است. 
\end{enumerate}




\begin{figure}[H]
	\begin{center}
		\includegraphics[width=.80\textwidth]{images/sensors-3.jpg}
	\end{center}
	\caption{اتصال تمامی سنسورها}
\end{figure}

هم چنین در این فاز مجددا سنسور 
\lr{ECG}
را تست کردیم و این بار توانستیم خروجی دقیق‌تری برای نوار قلب به دست آوریم که در شکل زیر قابل مشاهده است.

\begin{figure}[H]
	\begin{center}
		\includegraphics[width=.80\textwidth]{images/ecg-1.png}
	\end{center}
	\caption{
نوار قلب ثبت شده
}
\end{figure}


\subsubsection{سنسور فشار}

تنها سنسور باقی‌مانده از سنسورهای بدن که تا قبل از این فاز راه‌اندازی و تست نشد، سنسور فشار بوده است. در این فاز با استفاده از ماژول \lr{MPS20N0040D} و با کمک قطعه‌ی \lr{LM358} که شامل دو آپ‌امپ است و جهت تقویت خروجی سنسور به کار می‌رود، مداری به شکل زیر جهت گرفتن فشار خون بیمار بستیم. هم چنین کاف، فشارسنج و گوشی پزشکی را تهیه کردیم تا بتوانیم فشار خون بیمار را با استفاده از کاف پزشکی بگیریم و مدار بسته شده بگیریم؛ اما در نهایت موفق به گرفتن خروجی معنی‌دار از سنسور نشدیم. 


\begin{figure}[H]
	\begin{center}
		\includegraphics[width=.80\textwidth]{images/pressure-3.png}
	\end{center}
	\caption{شماتیک مدار مربوط به سنسور فشار}
\end{figure}

\begin{figure}[H]
	\begin{center}
		\includegraphics[width=.80\textwidth]{images/pressure-1.jpg}
	\end{center}
	\caption{مدار بسته‌ شده‌ی سنسور فشار}
\end{figure}


با توجه به زمان زیادی که صرف این قسمت شد و این که حتی در صورت اتصال درست این سنسور، برای گرفتن فشار خون بیمار نیاز به یک عامل انسانی وجود دارد که با طریقه‌ی گرفتن فشار آشنا باشد و با استفاده از گوشی پزشکی زمان ثبت فشار سیستولیک و دیاستولیک را در سیستم مشخص کند، تصمیم گرفتیم به جای این کار در صفحه‌ی نمایش یک جایگاه ورودی برای وارد کردن مقدار فشار در زمان‌های مختلف قرار دهیم تا امکان مشاهده‌ی مقدار آن در کنار سایر علائم امکان‌پذیر باشد.


\begin{figure}[H]
	\begin{center}
		\includegraphics[width=.80\textwidth]{images/cuff-2.jpg}
	\end{center}
	\caption{کاف، فشارسنج و گوشی پزشکی}
\end{figure}

\begin{figure}[H]
	\begin{center}
		\includegraphics[width=.80\textwidth]{images/sensors-pressure-3.jpg}
	\end{center}
	\caption{سنسور فشار در کنار سایر اجزای سیستم}
\end{figure}



\subsection{سرور}

در این فاز سمت کد سرور تغییری نداشتیم. کد اتصال آردویینو به رزبری و رزبری به سرور تغییراتی داشت تا امکان جمع‌آوری اطلاعات از همه‌ی سنسورهای متصل به آردویینو به طور هم‌زمان و انتقال آن‌ها توسط رزبری به سرور را فراهم کند.

در حال حاضر آردویینو در بازه‌ها‌ی زمانی مشخص و کوتاه مدت اطلاعات را از سنسورهای متصل به خودش جمع‌آوری می‌کند. یک پردازه روی رزبری این اطلاعات را می‌خواند و اطلاعات مربوط به هر سنسور را در فایل مربوط به خودش ذخیره می‌کند. پس از آن پردازه‌ی دیگری این اطلاعات را از روی فایل‌ها و هم چنین اطلاعات سنسورهای متصل به رزبری را به طور مستقیم دریافت می‌کند و به سرور ارسال می‌کند.

\subsection{نرم‌افزار موبایل}
هدف اصلی در این فاز به‌طور کلی بهبود واسط کاربری اپ موبایل و اضافه کردن نمودارهای مفید بود. در صفحه‌ی بیمار (تصویر
\ref{patients_screen})
ظاهر کارت نمایش‌دهنده‌ی آخرین اطلاعات بیمار تغییرات گسترده‌ای داشته تا به خوانایی و زیبایی این صفحه کمک کند. همچنین در صفحه‌ی نمودارها (تصویر 
\ref{old_charts})
در همین راستا به نمودارهای قبلی یک راهنما اضافه شده و همچنین دو نمودار جدید (نمایش داده شده در تصویر
\ref{new_charts})
اضافه شده که با استفاده از 
\lr{scatter plot}
به‌ترتیب، ارتباط آلودگی هوا و رطوبت نسبی (محور افقی) را با میزان اکسیژن خون (محور عمودی) می‌سنجند. همچنین سایز هر نقطه در این نمودارها نشان‌گر دمای  بدن بیمار در آن رکورد است.

\begin{figure} % [H]
	\begin{center}
		\begin{subfigure}{.3\textwidth}
			\includegraphics[width=.9\linewidth]{app_patients}
			\caption{صفحه بیمار}
			\label{patients_screen}
		\end{subfigure}
		\begin{subfigure}{.3\textwidth}
			\includegraphics[width=.9\linewidth]{app_charts1}
			\caption{نمودارهای قبلی}
			\label{old_charts}
		\end{subfigure}
		\begin{subfigure}{.3\textwidth}
			\includegraphics[width=.9\linewidth]{app_charts2}
			\caption{نمودارهای جدید}
			\label{new_charts}
		\end{subfigure}
		\caption{تصاویری از محیط نرم‌افزار موبایل}
		\label{app_screenshots}
	\end{center}
\end{figure}

\newpage
\section{زمان‌بندی} \label{gantt}

\subsection{چارت زمانی}

\begin{figure}[H]
	
	\begin{center}
		\begin{ganttchart}[
			expand chart=1\textwidth,
		    vrule label font=\tiny,
			title label font=\tiny, 
			bar label font=\tiny, 
			y unit title=1cm,
			y unit chart=0.8cm,
			x unit=1cm,
			vgrid,hgrid, 
			title label anchor/.style={below=-1.6ex},
			title left shift=0,
			title right shift=0,
			title height=1,
			progress label text={},
			bar height=0.6,
			group right shift=0,
			group top shift=.5,
			group height=.2]{3}{16}
			%labels
			\gantttitle{اسفند}{2}
			\gantttitle{فروردین}{4}
			\gantttitle{اردی‌بهشت}{4}
			\gantttitle{خرداد}{4}
			\\

			\gantttitle{\rl{هفته ۳}}{1} 			
			\gantttitle{\rl{هفته ۴}}{1} 
			\gantttitle{\rl{هفته ۱}}{1} 
			\gantttitle{\rl{هفته ۲}}{1} 
			\gantttitle{\rl{هفته ۳}}{1} 
			\gantttitle{\rl{هفته ۴}}{1} 
			\gantttitle{\rl{هفته ۱}}{1} 
			\gantttitle{\rl{هفته ۲}}{1} 
			\gantttitle{\rl{هفته ۳}}{1} 
			\gantttitle{\rl{هفته ۴}}{1} 
			\gantttitle{\rl{هفته ۱}}{1} 
			\gantttitle{\rl{هفته ۲}}{1} 
			\gantttitle{\rl{هفته ۳}}{1} 
			\gantttitle{\rl{هفته ۴}}{1} 

			%
			\\
			%tasks
			\ganttbar[progress=100]{\rl{نهایی کردن پروپوزال}}{3}{4} \\
			\ganttbar[progress=100]{\rl{تهیه‌ی قطعات}}{4}{6} \\
			\ganttbar[progress=100]{\rl{آشنایی با رزبری}}{5}{6} \\
			\ganttbar[progress=100]{\rl{تهیه معماری کامل سیستم}}{5}{6} \\
			
			\ganttgroup{\rl{سرور}}{6}{13} \\
			\ganttbar[progress=100]{\rl{بیسیک سرور}}{6}{7} \\	
			\ganttbar[progress=100]{\rl{اتصال رزبری و سرور}}{9}{10} \\
			\ganttbar[progress=75]{\rl{تکمیل سرور}}{10}{13} \\
			
			\ganttgroup{\rl{اپ موبایل}}{6}{13} \\
			\ganttbar[progress=100]{\rl{بیسیک اپ موبایل}}{6}{7} \\
			\ganttbar[progress=100]{\rl{اتصال اپ موبایل به سرور}}{8}{8} \\
			\ganttbar[progress=80]{\rl{رسم نمودارها و قابلیت‌های اضافه}}{9}{13} \\

			\ganttgroup{\rl{سخت‌افزار}}{7}{12} \\
			\ganttbar[progress=100]{\rl{اتصال و تست صفحه نمایش}}{7}{8} \\
			\ganttbar[progress=100]{\rl{اتصال و تست سنسورهای محیطی}}{7}{8} \\
			\ganttbar[progress=100]{\rl{اتصال و تست سنسورهای بدن}}{9}{12} \\
			
			\ganttbar[progress=0]{\rl{تست کلی سیستم}}{13}{14} \\
			
			\ganttbar[progress=0]{\rl{بررسی بازخوردها}}{15}{15}
			
			%relations 
			\ganttlink{elem0}{elem1} 
			\ganttlink{elem0}{elem3} 
			\ganttlink{elem0}{elem4} 
			\ganttlink{elem0}{elem8} 
			\ganttlink{elem1}{elem12}
			\ganttlink{elem3}{elem12} 			 
			\ganttlink{elem4}{elem16} 
			\ganttlink{elem8}{elem16}
			\ganttlink{elem12}{elem16} 
			\ganttlink{elem16}{elem17} 
			
			\ganttlink{elem5}{elem6} 
			\ganttlink{elem6}{elem7} 

			\ganttlink{elem5}{elem10}	
			
			\ganttlink{elem9}{elem10}	
			\ganttlink{elem10}{elem11} 
 
			\ganttvrule[vrule/.append style={blue, thin},vrule offset=1]{\rl{پروپوزال}}{4}
			\ganttvrule[vrule/.append style={blue, thin},vrule offset=1]{\rl{گزارش اول}}{6}
			\ganttvrule[vrule/.append style={blue, thin},vrule offset=1]{\rl{گزارش دوم}}{8}
			\ganttvrule[vrule/.append style={blue, thin},vrule offset=1]{\rl{گزارش سوم}}{10}
			\ganttvrule[vrule/.append style={blue, thin},vrule offset=1]{\rl{گزارش چهارم}}{12}	
			\ganttvrule[vrule/.append style={blue, thin},vrule offset=1]{\rl{گزارش اولیه}}{14}
			\ganttvrule[vrule/.append style={blue, thin},vrule offset=1]{\rl{گزارش نهایی}}{15}
		\end{ganttchart}
	\end{center}
	\caption{گانت چارت پروژه}
	
\end{figure}





\end{document}

