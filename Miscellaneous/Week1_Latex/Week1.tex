\documentclass[12pt]{article}
\usepackage{graphicx,import}
\usepackage{float}
\usepackage[svgnames]{xcolor} 
\usepackage{makecell}
\usepackage{fancyhdr}
\usepackage{subcaption}
\usepackage{hyperref}
\usepackage{enumitem}
\usepackage{cite}
\usepackage[many]{tcolorbox}
\usepackage{listings }
\usepackage[a4paper, total={6in, 8in} , bottom = 25mm , top = 25mm, headheight = 1.25cm , includehead,includefoot,heightrounded ]{geometry}
\usepackage{afterpage}
\usepackage{amssymb}
\usepackage{pdflscape}
\usepackage{gensymb}
\usepackage{textcomp}
\usepackage{tikz,pgfplots}
\usepackage{xecolor}
\usepackage{rotating}
\usepackage{pdfpages}
\usepackage[Kashida]{xepersian}
\usepackage[T1]{fontenc}
\usepackage{tikz}
\usepackage[utf8]{inputenc}
\usepackage{PTSerif} 
\usepackage{seqsplit}
\usepackage{hhline}
\usepackage{pgfgantt}
\usepackage{graphicx}

 
\graphicspath{ {./images/} }

\renewcommand\theadalign{bc}
\renewcommand\theadfont{\bfseries}
\renewcommand\theadgape{\Gape[4pt]}
\renewcommand\cellgape{\Gape[4pt]}

\usepackage[edges]{forest}

\usepackage{listings}
\usepackage{xcolor}

\hypersetup{
	colorlinks   = true, %Colours links instead of ugly boxes
	urlcolor     = blue, %Colour for external hyperlinks
	linkcolor    = blue, %Colour of internal links
	citecolor   = red %Colour of citations
}
 
\definecolor{codegreen}{rgb}{0,0.6,0}
\definecolor{codegray}{rgb}{0.5,0.5,0.5}
\definecolor{codepurple}{rgb}{0.58,0,0.82}
\definecolor{backcolour}{rgb}{0.95,0.95,0.92}
 
\NewDocumentCommand{\codeword}{v}{
\texttt{\textcolor{blue}{#1}}
}
\lstset{language=java,keywordstyle={\bfseries \color{blue}}}


\lstdefinestyle{mystyle}{
    backgroundcolor=\color{backcolour},   
    commentstyle=\color{codegreen},
    keywordstyle=\color{magenta},
    numberstyle=\tiny\color{codegray},
    stringstyle=\color{codepurple},
    basicstyle=\ttfamily\normalsize,
    breakatwhitespace=false,         
    breaklines=true,                 
    captionpos=b,                    
    keepspaces=true,                 
    numbers=left,                    
    numbersep=5pt,                  
    showspaces=false,                
    showstringspaces=false,
    showtabs=false,                  
    tabsize=2
}

\lstset{style=mystyle}

\settextfont[Scale=1.2 ,BoldFont={Bahij Nazanin-Bold} , ItalicFont = {IRNazanin}]{Bahij Nazanin-Regular}
\setlatintextfont[Scale = 1.0]{Garamond}
\DefaultMathsDigits 
\DeclareMathSizes{11}{19}{13}{9} 
%\DeclareMathSizes{12}{14.4}{8}{9}





\newenvironment{changemargin}[2]{%
\begin{list}{}{%
\setlength{\topsep}{0pt}%
\setlength{\leftmargin}{#1}%
\setlength{\rightmargin}{#2}%
\setlength{\listparindent}{\parindent}%
\setlength{\itemindent}{\parindent}%
\setlength{\parsep}{\parskip}%
}%
\item[]}{\end{list}}


\definecolor{foldercolor}{RGB}{124,166,198}

\tikzset{pics/folder/.style={code={%
    \node[inner sep=0pt, minimum size=#1](-foldericon){};
    \node[folder style, inner sep=0pt, minimum width=0.3*#1, minimum height=0.6*#1, above right, xshift=0.05*#1] at (-foldericon.west){};
    \node[folder style, inner sep=0pt, minimum size=#1] at (-foldericon.center){};}
    },
    pics/folder/.default={20pt},
    folder style/.style={draw=foldercolor!80!black,top color=foldercolor!40,bottom color=foldercolor}
}

\forestset{is file/.style={edge path'/.expanded={%
        ([xshift=\forestregister{folder indent}]!u.parent anchor) |- (.child anchor)},
        inner sep=1pt},
    this folder size/.style={edge path'/.expanded={%
        ([xshift=\forestregister{folder indent}]!u.parent anchor) |- (.child anchor) pic[solid]{folder=#1}}, inner xsep=0.6*#1},
    folder tree indent/.style={before computing xy={l=#1}},
    folder icons/.style={folder, this folder size=#1, folder tree indent=3*#1},
    folder icons/.default={12pt},
}

\begin{document}


%%% title pages
\begin{titlepage}
\begin{center}
        
\vspace*{0.7cm}

\includegraphics[width=0.4\textwidth]{sharif1.png}\\
\vspace{0.5cm}
\textbf{ \Huge{\emph ‌آزمایشگاه سخت‌افزار} }\\
\vspace{0.5cm}
\textbf{ \Large{گزارش فاز اول} }
\vspace{0.2cm}
       
 
      \large \textbf{دانشکده مهندسی کامپیوتر}\\\vspace{0.2cm}
    \large   دانشگاه صنعتی شریف\\\vspace{0.2cm}
       \large   ﻧﯿﻢ سال دوم 01-00 \\\vspace{0.2cm}
      \noindent\rule[1ex]{\linewidth}{1pt}
استاد:\\
    \textbf{{جناب آقای دکتر اجلالی}}


دستیار آموزشی:\\
\textbf{{جناب آقای دکتر فصحتی}}

    \vspace{0.25cm}
    
    موضوع پروژه:\\
    
    \textbf{{نمایشگر علائم حیاتی بیمار (پروژه شماره ۱۴)}}
    
    \vspace{0.35cm}
    
    
        شماره گروه:
    \textbf{{۲}}\\
    
اعضای گروه:\\

    \textbf{{علیرضا تاج‌میرریاحی - 97101372}}
    \\
   
     \textbf{{امیرمهدی نامجو - 97107212}}   
   \\
   
    \textbf{{ صبا هاشمی - 97100581}}
\end{center}
\end{titlepage}
%%% title pages


%%% header of pages
\newpage
\pagestyle{fancy}
\fancyhf{}
\fancyfoot{}
\cfoot{\thepage}
\chead{}
\rhead{\includegraphics[width=0.1\textwidth]{sharif.png}}
\lhead{گزارش فاز اول}
%%% header of pages

\newfontfamily\terminal{Courier New Bold}

\KashidaOff
 \newcommand{\inlineLatin}[1]{
	\small{\lr{{\terminal #1}}}
}


\tableofcontents
\listoffigures
% \listoftables

\newpage
\section{مقدمه}


محصول نهایی این پروژه، یک سیستم نمایشگر هوشمند علائم حیاتی بیمار و شرایط محیطی است. هسته این سیستم که از رزبری پای تشکیل شده است، اطلاعات حیاتی بیمار شامل دمای بدن، فشار خون، ضربان قلب، اکسیژن خون و نوار قلب (\lr{ECG}) را از طریق سنسور‌های مربوطه از بیمار دریافت کرده و در کنار آن، اطلاعات محیطی نظیر دما،‌ رطوبت و میزان آلودگی هوا را هم از طریق سنسورهایی دیگر دریافت می‌کند.

طبق زمان‌بندی ارائه شده در بخش \ref{gantt}، اقدامات مربوط به فاز اول پروژه عبارت‌اند از تهیه قطعات، آشنایی با رزبری، تهیه معماری سیستم و طراحی بیسیک سرور و اپ موبایل.
در ادامه به ارائه گزارش هریک از اقدامات فوق و نتایج آن‌ها خواهیم پرداخت.

\section{گزارش انجام پروژه}
\subsection{تهیه قطعات}

لیست قطعات تهیه شده به همراه قیمت تخمینی و قیمت تهیه‌شده در جدول زیر قابل مشاهده است:

\begin{table}[h]
	\begin{tabular}{|c|c|c|c|}
		\hline
		\textbf{ردیف} &
		\textbf{قطعه} &
		\textbf{\begin{tabular}[c]{@{}c@{}}قیمت کل\\  تخمین‌زده شده\\ (هزار تومان)\end{tabular}} &
		\textbf{\begin{tabular}[c]{@{}c@{}}قیمت کل نهایی\\ (هزار تومان)\end{tabular}} \\ \hline
		۱  & سنسور کیفیت و آلودگی هوا \lr{MQ-135} & \lr{38.3}   & \lr{38.3*}  \\ \hline
		۲ &
		\begin{tabular}[c]{@{}c@{}}سنسور اندازه‌گیری فشار خون\\ \lr{MPS20N0040D}\end{tabular} &
		\lr{26.5} &
		\lr{26.5} \\ \hline
		۳  & سنسور دما و رطوبت هوا \lr{KY-015}    & \lr{45}     & \lr{45*}    \\ \hline
		۴ &
		\begin{tabular}[c]{@{}c@{}}سنسور دمای بدن\\  \lr{MAX30205}\end{tabular} &
		\lr{290.9} &
		\lr{290.9} \\ \hline
		۵  & سنسور ECG ضربان قلب \lr{AD8232}      & \lr{128}    & \lr{128}    \\ \hline
		۶  & الکترود ECG ضربان قلب                & \lr{96}     & \lr{96}     \\ \hline
		۷  & ماژول اکسیمتر \lr{MAX30102}          & \lr{163.3}  & \lr{67}     \\ \hline
		۸  & صفحه نمایش LCD رزبری‌پای             & \lr{1480}   & \lr{1480*}  \\ \hline
		۹  & رزبری‌پای \lr{B3}                    & \lr{3000}   & \lr{3000*}  \\ \hline
		۱۰ & مبدل آنالوگ به دیجیتال \lr{ADS1115}  & \lr{154}    & \lr{154*}   \\ \hline
		۱۱ & بردبورد                              & \lr{47.5}   & \lr{47.5*}  \\ \hline
		۱۲ & سیم جامپر                            & \lr{100}    & \lr{100*}   \\ \hline
		۱۳ & پد ECG                               & \lr{12}     & \lr{12}     \\ \hline
		۱۴ & مقاومت                               & \lr{7}      & \lr{7}      \\ \hline
		& \textbf{مجموع}                       & \lr{5588.5} & \lr{5492.5} \\ \hline
	\end{tabular}
	\caption{هزینه‌ها}
\end{table}

قطعاتی که کنار قیمت نهایی آن‌ها علامت * وجود دارد، به علت موجود بودن خریداری نشده‌اند و قیمت نهایی آن‌ها همان قیمت تخمینی آن‌هاست و قیمت نهایی برای مقایسه‌ی هزینه نهایی پروژه با هزینه‌ی تخمین‌زده شده آورده شده‌ است.


هزینه‌ی تخمین‌زده شده‌ی ما از این پروژه ۵ میلیون و ۵۸۸ هزار تومان بوده است و هزینه‌ی نهایی برابر ۵ میلیون و ۴۹۲ هزار تومان شد. از این مقدار قطعاتی به ارزش تخمینی ۴ میلیون و ۷۵۰ هزار تومان از آزمایشگاه تهیه شد و سایر قطعات به قیمت ۷۴۰ هزار تومان جداگانه خریداری شد.

در زیر تصویر قطعات تهیه شده قابل مشاهده است:
TODO


\subsection{آشنایی با رزبری}
\subsection{معماری سیستم}
معماری اولیه شامل ساختار کلی و سطح بالای سیستم و معرفی ابزارها و نحوه‌ی اتصال آن‌ها که در پروپوزال پروژه معرفی شد مطابق زیر است:
\begin{figure}[h]
	\begin{center}
		\includegraphics[width=.85\textwidth,trim={0 14cm 0 1cm},clip]{initial_arch} % left bottom right top
	\end{center}
	\caption{معماری سطح بالای سیستم}
\end{figure}

در ادامه‌ی این بخش جزئیات معماری هر قسمت از سیستم به‌همراه نیازمندی‌ها معرفی می‌شود.

\subsubsection{نیازمندی‌های سیستم}
جهت تعیین معماری اجزای سیستم ابتدا لازم است عملکرد مورد نیاز از سیستم به‌طور دقیق‌تری بررسی شود:
\begin{itemize}
	\item سیستم نام کاربر را دریافت و ذخیره کند و امکان تغییر آن وجود داشته باشد.
	\item امکان اعلام شروع و پایان فرایند ثبت داده وجود داشته باشد.
	\item اطلاعات حیاتی و شرایط محیطی به‌صورت متناوب هر دقیقه یک بار به سرور ارسال گردد.
	\item با توجه به پیوسته نبودن جمع‌آوری داده در طول زمان در برخی سنسورها، ذخیره و تمیز کردن داده‌ها مدیریت شود.
	\item امکان مصورسازی داده‌های دوره‌ای به‌ازای کاربر در بازه‌های مختلف وجود داشته باشد.
	\item امکان مصورسازی داده‌ها جهت بررسی هم‌بستگی علائم حیاتی و شرایط محیطی وجود داشته باشد.
	\item در صورت مشاهده‌ی مقادیر غیر عادی به کاربر هشدار دهد.
\end{itemize}

\subsubsection{شمای کلی}
اجزای سیستم و نحوه‌ی ارتباط آن‌ها در تصویر زیر مشخص شده است:
\begin{figure}[h]
	\begin{center}
		\includegraphics[width=\textwidth,trim={1cm 4cm 1cm 4cm},clip]{project_architecture}
	\end{center}
	\caption{شمای کلی سیستم}
\end{figure}

\subsection{طراحی بیسیک سرور}
نسخه‌ی اولیه‌ی سرور با کمک فریم‌ورک \lr{Django} ایجاد شده و به‌وسیله‌ی ابزار Docker استقرار می‌یابد. همچنین با کمک \lr{docker compose} و یک تصویر از دیتابیس \lr{MySQL}، اتصال سرور به دیتابیس برقرار می‌شود. 

در حال حاضر دو مدل \lr{Record} و \lr{Patient} در سرور تعریف شده که برای ذخیره‌ی اطلاعات کاربران و داده‌ی سنسورها مورد استفاده قرار خواهد گرفت.

همچنین مستندسازی \lr{API} سرور با شمای \lr{Redoc} و \lr{Swagger} فراهم شده که پس از اجرای سرور به‌ترتیب در مسیر \codeword{/redoc} و \codeword{/swagger} قابل مشاهده خواهد بود. تصویر بخشی از مستند در صفحه‌ی \lr{redoc} در شکل \ref{redoc} نشان داده شده است.

\begin{figure}[h]
	\begin{center}
		\includegraphics[width=\textwidth]{redoc}
	\end{center}
	\caption{نمونه‌ی مستندات \lr{API} سرور}
	\label{redoc}
\end{figure}

کد برنامه در مخزن پروژه در 
\href{https://github.com/Sharif-University-ESRLab/project-team-2/tree/main/Code/backend}{این مسیر}
قرار گرفته است.

\subsection{طراحی بیسیک اپ موبایل}
نسخه‌ی اولیه‌ی اپ موبایل با کمک ابزار \lr{React Native} ایجاد شده است و در حال حاضر قابلیت مشاهده‌ی لیست بیماران با اتصال به سرور فراهم شده است.
در شکل
 \ref{app_screenshots}
، تصاویری از محیط برنامه به کمک \lr{Expo} اجرا شده قابل مشاهده است.

\begin{figure}[H]
	\begin{center}
		\begin{subfigure}{.45\textwidth}
			\includegraphics[width=.9\linewidth]{app_home}
			\caption{صفحه اصلی}
		\end{subfigure}
		\begin{subfigure}{.45\textwidth}
			\includegraphics[width=.9\linewidth]{app_patients}
			\caption{لیست بیماران ثبت شده}
		\end{subfigure}
		\caption{تصاویری از محیط اپ موبایل}
		\label{app_screenshots}
	\end{center}
\end{figure}

کد برنامه و نحوه‌ی نصب و راه‌اندازی آن در 
\href{https://github.com/Sharif-University-ESRLab/project-team-2/tree/main/Code/app}{این مسیر}
قرار گرفته است.

\newpage
\section{زمان‌بندی} \label{gantt}

\subsection{چارت زمانی}

\begin{figure}[H]
	
	\begin{center}
		\begin{ganttchart}[
			expand chart=1\textwidth,
		    vrule label font=\tiny,
			title label font=\tiny, 
			bar label font=\tiny, 
			y unit title=1cm,
			y unit chart=0.8cm,
			x unit=1cm,
			vgrid,hgrid, 
			title label anchor/.style={below=-1.6ex},
			title left shift=0,
			title right shift=0,
			title height=1,
			progress label text={},
			bar height=0.6,
			group right shift=0,
			group top shift=.5,
			group height=.2]{3}{16}
			%labels
			\gantttitle{اسفند}{2}
			\gantttitle{فروردین}{4}
			\gantttitle{اردی‌بهشت}{4}
			\gantttitle{خرداد}{4}
			\\

			\gantttitle{\rl{هفته ۳}}{1} 			
			\gantttitle{\rl{هفته ۴}}{1} 
			\gantttitle{\rl{هفته ۱}}{1} 
			\gantttitle{\rl{هفته ۲}}{1} 
			\gantttitle{\rl{هفته ۳}}{1} 
			\gantttitle{\rl{هفته ۴}}{1} 
			\gantttitle{\rl{هفته ۱}}{1} 
			\gantttitle{\rl{هفته ۲}}{1} 
			\gantttitle{\rl{هفته ۳}}{1} 
			\gantttitle{\rl{هفته ۴}}{1} 
			\gantttitle{\rl{هفته ۱}}{1} 
			\gantttitle{\rl{هفته ۲}}{1} 
			\gantttitle{\rl{هفته ۳}}{1} 
			\gantttitle{\rl{هفته ۴}}{1} 

			%
			\\
			%tasks
			\ganttbar[progress=100]{\rl{نهایی کردن پروپوزال}}{3}{4} \\
			\ganttbar[progress=100]{\rl{تهیه‌ی قطعات}}{4}{6} \\
			\ganttbar[progress=100]{\rl{آشنایی با رزبری}}{5}{6} \\
			\ganttbar[progress=100]{\rl{تهیه معماری کامل سیستم}}{5}{6} \\
			
			\ganttgroup{\rl{سرور}}{6}{13} \\
			\ganttbar[progress=50]{\rl{بیسیک سرور}}{6}{7} \\	
			\ganttbar[progress=0]{\rl{اتصال رزبری و سرور}}{8}{9} \\
			\ganttbar[progress=0]{\rl{تکمیل سرور}}{10}{13} \\
			
			\ganttgroup{\rl{اپ موبایل}}{6}{13} \\
			\ganttbar[progress=50]{\rl{بیسیک اپ موبایل}}{6}{7} \\
			\ganttbar[progress=0]{\rl{اتصال اپ موبایل به سرور}}{8}{8} \\
			\ganttbar[progress=0]{\rl{رسم نمودارها و قابلیت‌های اضافه}}{9}{13} \\

			\ganttgroup{\rl{سخت‌افزار}}{7}{12} \\
			\ganttbar[progress=0]{\rl{اتصال و تست صفحه نمایش}}{7}{8} \\
			\ganttbar[progress=0]{\rl{اتصال و تست سنسورهای محیطی}}{7}{8} \\
			\ganttbar[progress=0]{\rl{اتصال و تست سنسورهای بدن}}{9}{12} \\
			
			\ganttbar[progress=0]{\rl{تست کلی سیستم}}{13}{14} \\
			
			\ganttbar[progress=0]{\rl{بررسی بازخوردها}}{15}{15}
			
			%relations 
			\ganttlink{elem0}{elem1} 
			\ganttlink{elem0}{elem3} 
			\ganttlink{elem0}{elem4} 
			\ganttlink{elem0}{elem8} 
			\ganttlink{elem1}{elem12}
			\ganttlink{elem3}{elem12} 			 
			\ganttlink{elem4}{elem16} 
			\ganttlink{elem8}{elem16}
			\ganttlink{elem12}{elem16} 
			\ganttlink{elem16}{elem17} 
			
			\ganttlink{elem5}{elem6} 
			\ganttlink{elem6}{elem7} 

			\ganttlink{elem5}{elem10}	
			
			\ganttlink{elem9}{elem10}	
			\ganttlink{elem10}{elem11} 
 
			\ganttvrule[vrule/.append style={blue, thin},vrule offset=1]{\rl{پروپوزال}}{4}
			\ganttvrule[vrule/.append style={blue, thin},vrule offset=1]{\rl{گزارش اول}}{6}
			\ganttvrule[vrule/.append style={blue, thin},vrule offset=1]{\rl{گزارش دوم}}{8}
			\ganttvrule[vrule/.append style={blue, thin},vrule offset=1]{\rl{گزارش سوم}}{10}
			\ganttvrule[vrule/.append style={blue, thin},vrule offset=1]{\rl{گزارش چهارم}}{12}	
			\ganttvrule[vrule/.append style={blue, thin},vrule offset=1]{\rl{گزارش اولیه}}{14}
			\ganttvrule[vrule/.append style={blue, thin},
vrule offset=1
]{\rl{گزارش نهایی}}{15}
		\end{ganttchart}
	\end{center}
	\caption{گانت چارت پروژه}
	
\end{figure}
	
\end{document}



