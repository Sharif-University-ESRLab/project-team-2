\documentclass[12pt]{article}
\usepackage{graphicx,import}
\usepackage{float}
\usepackage[svgnames]{xcolor} 
\usepackage{makecell}
\usepackage{fancyhdr}
\usepackage{subcaption}
\usepackage{hyperref}
\usepackage{enumitem}
\usepackage[many]{tcolorbox}
\usepackage{listings }
\usepackage[a4paper, total={6in, 8in} , bottom = 25mm , top = 25mm, headheight = 1.25cm , includehead,includefoot,heightrounded ]{geometry}
\usepackage{afterpage}
\usepackage{amssymb}
\usepackage{pdflscape}
\usepackage{gensymb}
\usepackage{textcomp}
\usepackage{tikz,pgfplots}
\usepackage{xecolor}
\usepackage{rotating}
\usepackage{pdfpages}
\usepackage[Kashida]{xepersian}
\usepackage[T1]{fontenc}
\usepackage{tikz}
\usepackage[utf8]{inputenc}
\usepackage{PTSerif} 
\usepackage{seqsplit}
\usepackage{hhline}
\usepackage{pgfgantt}
\usepackage{graphicx}
\usepackage{filecontents}
\usepackage{url} % for "\url" macro
\usepackage{babel}
\usepackage[backend=bibtex, style=numeric]{biblatex}


\graphicspath{ {./images/} }

\renewcommand\theadalign{bc}
\renewcommand\theadfont{\bfseries}
\renewcommand\theadgape{\Gape[4pt]}
\renewcommand\cellgape{\Gape[4pt]}

\usepackage[edges]{forest}

\usepackage{listings}
\usepackage{xcolor}

\hypersetup{
	colorlinks   = true, %Colours links instead of ugly boxes
	urlcolor     = blue, %Colour for external hyperlinks
	linkcolor    = blue, %Colour of internal links
	citecolor   = red %Colour of citations
}
 
\definecolor{codegreen}{rgb}{0,0.6,0}
\definecolor{codegray}{rgb}{0.5,0.5,0.5}
\definecolor{codepurple}{rgb}{0.58,0,0.82}
\definecolor{backcolour}{rgb}{0.95,0.95,0.92}
 
\NewDocumentCommand{\codeword}{v}{
\texttt{\textcolor{blue}{#1}}
}
\lstset{language=java,keywordstyle={\bfseries \color{blue}}}


\lstdefinestyle{mystyle}{
    backgroundcolor=\color{backcolour},   
    commentstyle=\color{codegreen},
    keywordstyle=\color{magenta},
    numberstyle=\tiny\color{codegray},
    stringstyle=\color{codepurple},
    basicstyle=\ttfamily\normalsize,
    breakatwhitespace=false,         
    breaklines=true,                 
    captionpos=b,                    
    keepspaces=true,                 
    numbers=left,                    
    numbersep=5pt,                  
    showspaces=false,                
    showstringspaces=false,
    showtabs=false,                  
    tabsize=2
}

\lstset{style=mystyle}

\settextfont[Scale=1.2 ,BoldFont={Bahij Nazanin-Bold} , ItalicFont = {IRNazanin}]{Bahij Nazanin-Regular}
\setlatintextfont[Scale = 1.0]{Garamond}
\DefaultMathsDigits 
\DeclareMathSizes{11}{19}{13}{9} 
%\DeclareMathSizes{12}{14.4}{8}{9}





\newenvironment{changemargin}[2]{%
\begin{list}{}{%
\setlength{\topsep}{0pt}%
\setlength{\leftmargin}{#1}%
\setlength{\rightmargin}{#2}%
\setlength{\listparindent}{\parindent}%
\setlength{\itemindent}{\parindent}%
\setlength{\parsep}{\parskip}%
}%
\item[]}{\end{list}}


\definecolor{foldercolor}{RGB}{124,166,198}

\tikzset{pics/folder/.style={code={%
    \node[inner sep=0pt, minimum size=#1](-foldericon){};
    \node[folder style, inner sep=0pt, minimum width=0.3*#1, minimum height=0.6*#1, above right, xshift=0.05*#1] at (-foldericon.west){};
    \node[folder style, inner sep=0pt, minimum size=#1] at (-foldericon.center){};}
    },
    pics/folder/.default={20pt},
    folder style/.style={draw=foldercolor!80!black,top color=foldercolor!40,bottom color=foldercolor}
}

\forestset{is file/.style={edge path'/.expanded={%
        ([xshift=\forestregister{folder indent}]!u.parent anchor) |- (.child anchor)},
        inner sep=1pt},
    this folder size/.style={edge path'/.expanded={%
        ([xshift=\forestregister{folder indent}]!u.parent anchor) |- (.child anchor) pic[solid]{folder=#1}}, inner xsep=0.6*#1},
    folder tree indent/.style={before computing xy={l=#1}},
    folder icons/.style={folder, this folder size=#1, folder tree indent=3*#1},
    folder icons/.default={12pt},
}

\begin{document}


%%% title pages
\begin{titlepage}
\begin{center}
        
\vspace*{0.7cm}

\includegraphics[width=0.4\textwidth]{sharif1.png}\\
\vspace{0.5cm}
\textbf{ \Huge{\emph ‌آزمایشگاه سخت‌افزار} }\\
\vspace{0.5cm}
\textbf{ \Large{گزارش فاز سوم} }
\vspace{0.2cm}
       
 
      \large \textbf{دانشکده مهندسی کامپیوتر}\\\vspace{0.2cm}
    \large   دانشگاه صنعتی شریف\\\vspace{0.2cm}
       \large   ﻧﯿﻢ سال دوم 01-00 \\\vspace{0.2cm}
      \noindent\rule[1ex]{\linewidth}{1pt}
استاد:\\
    \textbf{{جناب آقای دکتر اجلالی}}


دستیار آموزشی:\\
\textbf{{جناب آقای دکتر فصحتی}}

    \vspace{0.25cm}
    
    موضوع پروژه:\\
    
    \textbf{{نمایشگر علائم حیاتی بیمار (پروژه شماره ۱۴)}}
    
    \vspace{0.35cm}
    
    
        شماره گروه:
    \textbf{{۲}}\\
    
اعضای گروه:\\

    \textbf{{علیرضا تاج‌میرریاحی - 97101372}}
    \\
   
     \textbf{{امیرمهدی نامجو - 97107212}}   
   \\
   
    \textbf{{ صبا هاشمی - 97100581}}
\end{center}
\end{titlepage}
%%% title pages


%%% header of pages
\newpage
\pagestyle{fancy}
\fancyhf{}
\fancyfoot{}
\cfoot{\thepage}
\chead{}
\rhead{\includegraphics[width=0.1\textwidth]{sharif.png}}
\lhead{گزارش فاز سوم}
%%% header of pages

\newfontfamily\terminal{Courier New Bold}

\KashidaOff
 \newcommand{\inlineLatin}[1]{
	\small{\lr{{\terminal #1}}}
}


\tableofcontents
\listoffigures

\newpage
\section{مقدمه}


محصول نهایی این پروژه، یک سیستم نمایشگر هوشمند علائم حیاتی بیمار و شرایط محیطی است. هسته این سیستم که از رزبری پای تشکیل شده است، اطلاعات حیاتی بیمار شامل دمای بدن، فشار خون، ضربان قلب، اکسیژن خون و نوار قلب (\lr{ECG}) را از طریق سنسور‌های مربوطه از بیمار دریافت کرده و در کنار آن، اطلاعات محیطی نظیر دما،‌ رطوبت و میزان آلودگی هوا را هم از طریق سنسورهایی دیگر دریافت می‌کند.

طبق زمان‌بندی ارائه شده در بخش \ref{gantt}، اقدامات مربوط به فاز سوم پروژه عبارت‌اند از اتصال و تست بخشی از سنسورهای بدن، اتصال رزبری به سرور، تکمیل بخشی از سرور و تکمیل بخشی از اپلیکیشن و اضافه کردن تمودارها و امکانات جانبی به آن. برای قسمت تکمیل سرور از آن جایی که بخش عمده آن در همان فازهای قبل تکمیل شده بود و در شرایط فعلی نیازی به قابلیت اضافه‌ای نداشتیم،‌ کار‌ی انجام نشد.




\section{گزارش انجام پروژه}
\subsection{سخت‌افزار}

در این بخش، به پیشرفت‌ها و چالش‌های زمینه راه‌اندازی قسمت‌های سخت‌افزاری پروژه، شامل اتصال و تست سنسورهای بدن \ref{body} می‌پردازیم.

\subsubsection{اتصال و تست سنسورهای بدن} \label{body}

برای سنسورهای مربوط به بدن، در این فاز سه سنسور زیر را راه‌اندازی کردیم.

\begin{enumerate}
	\item 
	سنسور \lr{Max30102}: این سنسور برای سنجش اکسیژن خون و ضربان قلب است.
	
	\item 
	سنسور \lr{MAX30205}: این سنسور برای سنجش دمای بدن استفاده می‌شود.
	
	\item 
	سنسور \lr{Ad8232}: این سنسور برای ECG استفاده می‌شود.
	
	
	
\end{enumerate}





\begin{itemize}
	\item \lr{\textbf{Max30102}}:
	
این سنسور با کمک دو چراغ کوچک قرمز و مادون قرمز، ضربان قلب و درصد اشباع اکسیژن در خون (\lr{SpO2}) را اندازه‌گیری می‌کند.

این سنسور بدون مشکل از طریق \lr{I2C} به رزبری‌پای متصل می‌شود. البته توجه کنید که همان‌طور که در گزارش قبلی گفته شد، این سنسور با رزبری‌پای \lr{3B} به خوبی کار نمی‌کند و در نتیجه از رزبری‌پای \lr{4} استفاده کردیم. 

	\begin{figure}[H]
		\begin{center}
			\includegraphics[width=.60\textwidth]{images/max30102-1.jpg}
		\end{center}
		\caption{اتصال سنسور \lr{Max30102}  به رزبری پای}
	\end{figure}

در زیر تصویری از خروجی آن را مشاهده می‌کنید:
	\begin{figure}[H]
	\begin{center}
		\includegraphics[width=.60\textwidth]{images/max30102-2.png}
	\end{center}
	\caption{خروجی سنسور \lr{Max30102}}
\end{figure}

\item \lr{\textbf{Max30205}}

کاربرد این سنسور برای سنجش دمای بدن است. با تماس انگشت به آن، بعد از مدتی دمای سنسور با دمای انگشت همدما شده و دمای بدن را نشان خواهد داد. در صورت عدم تماس هم می‌توان از آن برای مشاهده دمای محیط استفاده کرد.

این سسنور به آردوینو متصل شده و از طریق اتصال آردوینو به رزبری‌پای، اطلاعات آن را مشاهده می‌کنیم. در زیر تصویر اتصال این سنسور قرار دارد.

	\begin{figure}[H]
	\begin{center}
		\includegraphics[width=.60\textwidth]{images/max30205.jpg}
	\end{center}
	\caption{اتصال سنسور دمای بدن \lr{Max30205} از طریق آردوینو به رزبری‌پای}
\end{figure}


\item \lr{\textbf{Ad8232}}

از این سنسور برای بدست‌ آوردن \lr{ECG} (نوار قلب) استفاده می‌شود. همراه این سنسور بسته \lr{Lead} سه تایی اتصال به بدن وجود دارد. به این \lr{Lead} ها باید پد‌های مخصوص متصل شده و به بدن متصل شوند.

برای استفاده از سنسور،‌ آن را به آردوینو متصل کرده و آردوینو را به رزبری پای متصل می‌کنیم. تصویری که از نوار قلب بدست آوردیم، \lr{Peak}‌ های اصلی نوار قلب را به خوبی نشان می‌دهد ولی مقداری نویز در قسمت‌های دیگر دارد که احتمالا به دلیل نحوه اتصال Lead ها باشد. در تصویر زیر، برای تست سریع‌تر کد آردوینو مستقیما به لپ‌تاپ متصل شده است ولی نحوه انجام کار با رزبری هم تست شده و تفاوت خاصی ندارد چون عملا کد یکسانی باید روی آردوینو اجرا بشود.


\begin{figure}[H]
	\begin{center}
		\includegraphics[width=.60\textwidth]{images/ecg-1.jpg}
	\end{center}
	\caption{استفاده از سنسور \lr{Ad8232} برای بدست آوردن \lr{ECG} }
\end{figure}



\begin{figure}[H]
	\begin{center}
		\includegraphics[width=.60\textwidth]{images/ecg-2.png}
	\end{center}
	\caption{\lr{ECG} بدست آمده از خروجی سنسور. \lr{Peak}‌ های موجود در شکل درست هستند ولی سایر بخش‌ها مقداری نویز دارند. }
\end{figure}







\end{itemize}




\subsection{نرم‌افزار موبایل}




\newpage
\section{زمان‌بندی} \label{gantt}

\subsection{چارت زمانی}

\begin{figure}[H]
	
	\begin{center}
		\begin{ganttchart}[
			expand chart=1\textwidth,
		    vrule label font=\tiny,
			title label font=\tiny, 
			bar label font=\tiny, 
			y unit title=1cm,
			y unit chart=0.8cm,
			x unit=1cm,
			vgrid,hgrid, 
			title label anchor/.style={below=-1.6ex},
			title left shift=0,
			title right shift=0,
			title height=1,
			progress label text={},
			bar height=0.6,
			group right shift=0,
			group top shift=.5,
			group height=.2]{3}{16}
			%labels
			\gantttitle{اسفند}{2}
			\gantttitle{فروردین}{4}
			\gantttitle{اردی‌بهشت}{4}
			\gantttitle{خرداد}{4}
			\\

			\gantttitle{\rl{هفته ۳}}{1} 			
			\gantttitle{\rl{هفته ۴}}{1} 
			\gantttitle{\rl{هفته ۱}}{1} 
			\gantttitle{\rl{هفته ۲}}{1} 
			\gantttitle{\rl{هفته ۳}}{1} 
			\gantttitle{\rl{هفته ۴}}{1} 
			\gantttitle{\rl{هفته ۱}}{1} 
			\gantttitle{\rl{هفته ۲}}{1} 
			\gantttitle{\rl{هفته ۳}}{1} 
			\gantttitle{\rl{هفته ۴}}{1} 
			\gantttitle{\rl{هفته ۱}}{1} 
			\gantttitle{\rl{هفته ۲}}{1} 
			\gantttitle{\rl{هفته ۳}}{1} 
			\gantttitle{\rl{هفته ۴}}{1} 

			%
			\\
			%tasks
			\ganttbar[progress=100]{\rl{نهایی کردن پروپوزال}}{3}{4} \\
			\ganttbar[progress=100]{\rl{تهیه‌ی قطعات}}{4}{6} \\
			\ganttbar[progress=100]{\rl{آشنایی با رزبری}}{5}{6} \\
			\ganttbar[progress=100]{\rl{تهیه معماری کامل سیستم}}{5}{6} \\
			
			\ganttgroup{\rl{سرور}}{6}{13} \\
			\ganttbar[progress=100]{\rl{بیسیک سرور}}{6}{7} \\	
			\ganttbar[progress=0]{\rl{اتصال رزبری و سرور}}{9}{10} \\
			\ganttbar[progress=0]{\rl{تکمیل سرور}}{10}{13} \\
			
			\ganttgroup{\rl{اپ موبایل}}{6}{13} \\
			\ganttbar[progress=100]{\rl{بیسیک اپ موبایل}}{6}{7} \\
			\ganttbar[progress=100]{\rl{اتصال اپ موبایل به سرور}}{8}{8} \\
			\ganttbar[progress=0]{\rl{رسم نمودارها و قابلیت‌های اضافه}}{9}{13} \\

			\ganttgroup{\rl{سخت‌افزار}}{7}{12} \\
			\ganttbar[progress=100]{\rl{اتصال و تست صفحه نمایش}}{7}{8} \\
			\ganttbar[progress=100]{\rl{اتصال و تست سنسورهای محیطی}}{7}{8} \\
			\ganttbar[progress=0]{\rl{اتصال و تست سنسورهای بدن}}{9}{12} \\
			
			\ganttbar[progress=0]{\rl{تست کلی سیستم}}{13}{14} \\
			
			\ganttbar[progress=0]{\rl{بررسی بازخوردها}}{15}{15}
			
			%relations 
			\ganttlink{elem0}{elem1} 
			\ganttlink{elem0}{elem3} 
			\ganttlink{elem0}{elem4} 
			\ganttlink{elem0}{elem8} 
			\ganttlink{elem1}{elem12}
			\ganttlink{elem3}{elem12} 			 
			\ganttlink{elem4}{elem16} 
			\ganttlink{elem8}{elem16}
			\ganttlink{elem12}{elem16} 
			\ganttlink{elem16}{elem17} 
			
			\ganttlink{elem5}{elem6} 
			\ganttlink{elem6}{elem7} 

			\ganttlink{elem5}{elem10}	
			
			\ganttlink{elem9}{elem10}	
			\ganttlink{elem10}{elem11} 
 
			\ganttvrule[vrule/.append style={blue, thin},vrule offset=1]{\rl{پروپوزال}}{4}
			\ganttvrule[vrule/.append style={blue, thin},vrule offset=1]{\rl{گزارش اول}}{6}
			\ganttvrule[vrule/.append style={blue, thin},vrule offset=1]{\rl{گزارش دوم}}{8}
			\ganttvrule[vrule/.append style={blue, thin},vrule offset=1]{\rl{گزارش سوم}}{10}
			\ganttvrule[vrule/.append style={blue, thin},vrule offset=1]{\rl{گزارش چهارم}}{12}	
			\ganttvrule[vrule/.append style={blue, thin},vrule offset=1]{\rl{گزارش اولیه}}{14}
			\ganttvrule[vrule/.append style={blue, thin},vrule offset=1]{\rl{گزارش نهایی}}{15}
		\end{ganttchart}
	\end{center}
	\caption{گانت چارت پروژه}
	
\end{figure}





\end{document}

